\section{Stability and generalizability}
\subsection{Stability}
In order to measure the stability or robustness of the models they were evaluated on the corrupted or "bad" input DIC images. There are two sources of "bad" images that can be used for such estimations. One is the real corrupted images made in the laboratory. Corruptions in this cade may come from different sources: for example, bubble of oil that landed on the microscope lenses, low density of the cell on the image, over- or under-exposure. Another source would be images with pseudo corruptions created manually. 

In this subsection the results from evaluating the model on TODO types in image corruption are presented, namely: defocus blur, gaussian bluer, gaussian noise, .. TODO. Every corruption will have different effects on the prediction of the model based on the severity of the corruption. Therefore it is important to evaluate the error-rate (in this case a loss function) for the predictions for different severity levels of each of the corruption types. It is also important to take visual evaluation of the prediction qualities as well. Each of the corruptions $c$ will have 5 different severity levels $s$, meaning that $1 \leq s \leq 5$, however it is important to keep in mind that although the severity levels were chosen to be as much comparable between each other as possible, they still might have differences in the strengths, as for example brightness has much stronger effect on predictions than gaussian blur (TODO).

In the next subsections several artificial corruptions are presented.
\subsubsection{Gaussian Noise}
In order to intorduce Gaussian noise to the image one first can normalize the image between $0$ and $1$ and then add a Gaussian noise (\ref{eq:gaussian_noise}) with mean $0$ and standard deviation $\sigma$ to the normalized image, where $\sigma$ is a severity level. The resulting image is then denormalized to the original range. 

\begin{equation}
    \label{eq:gaussian_noise}
    \frac{1}{\sigma\sqrt{2\pi}} e^{-\frac{x^2}{2\sigma^2}}
\end{equation}

\subsubsection{Defocus Blur}
Defocus blur corruption imitates the effect of defocus on the microscope. The blur is applied to the image by convolving it with a special kernel. There are two tunable parameters for this corruption type: first one is the readius of the circle in the kernel $r$, and the second one is the blur strength parameter $s$. Examples of the kernel with radius $r$ is shown in the Figure below. Such kernel is then simply applied to an image via $cv2.filter2D$ function.

\subsubsection{Gaussian Blur}
Gaussian blur corruption is another type of blur corruption but with a Gaussian kernel. The center element of such a kernel is the largest one, it corresponds to the peak of Gaussian discribution. This value is descreasing along each side of the center element of the kernel in a symmetrical way. The blur strength parameter $s$ is the standard deviation of the Gaussian kernel. The kernel is again simply applied to an image and one recieves a blurred image as the result. For this corruption a $gaussian()$ function from $skimage.filters$ library was used.

\subsubsection{Brightness}
Different brightness levels are also an important image corruption to test on, which appears often in the dataset during image accquisition. In oder to change the brightness of the image, the image from the RGB format was translated into HSV format, which stands for hue, saturation and value. This is also one of popular formats to represent an image. To make an image brighter or darker, one can simply add or subtract a parameter $s$ in a value channel for each of the pixels correspondingly. This parameter is often called bias. The bigger absolute value of this change the stronger a corruption will be.

\begin{equation}
    \hat{x}_{i, j} = x_{i, j} + s
\end{equation}

\subsubsection{Contrast}
In contrast to adding a constant pixelwise to an image to chnage the brightness level, in order to change a contrast level one can perform a multiplication of an image with another constant $s$. This parameter is often called gain.

\begin{equation}
    \hat{x}_{i, j} = s * x_{i, j}
\end{equation}

For both contrast and brightness changes one can use $cv2.convertScaleAbs()$ which directly accepts gain and bias parameters and clips the image to stay within the allowed values range.

Here are presented the results of all the corruption types and their comparison. One can clearly notice that model is quite stable towards different brightness levels and contrast. Adding more contrast to an image even seems to slightly improve the predictions (as the loss becomes lower). 