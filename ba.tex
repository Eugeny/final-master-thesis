%%%%%%%%%%%%%%%%%%%%%%%%%%%%%%%%%%%%%%%%%%%%%%%%%%%%%%%%%%%%%%%%%%%%%%%%
% Uni Duesseldorf
% Lehrstuhl fuer Datenbanken und Informationssysteme
% Vorlage fuer Bachelor-/Masterarbeiten
% Optimiert fuer den Original-Latex-Kompiler LATEX.EXE (LaTeX=>PS=>PDF)
%%%%%%%%%%%%%%%%%%%%%%%%%%%%%%%%%%%%%%%%%%%%%%%%%%%%%%%%%%%%%%%%%%%%%%%%
% Ueberarbeitung für pdflatex (LaTeX=>PDF)
%%%%%%%%%%%%%%%%%%%%%%%%%%%%%%%%%%%%%%%%%%%%%%%%%%%%%%%%%%%%%%%%%%%%%%%%
% Vorlage Changelog:
% 10.09.2015 (Matthias Liebeck): Nummerierung des Inhaltsverzeichnis nun römisch, Beispiel für einen Anhang eingebaut, \raggedbottom hinter sections eingefügt
% 11.07.2018 (Matthias Liebeck): Ersetzung des Bibliographiestils, Einsatz von Biber
% 04.09.2018 (Matthias Liebeck):
%   * Bibtex: unnötige Bibtexfelder beim Rendern ausblenden (thx @ Markus Brenneis)
%   * ngerman: "et al." im BibTeX für drei oder mehr Autoren
%   * Neuer Befehl \sectionforcestartright: Sections immer rechts beginnen (thx @ Philipp Grawe)
%   * ngerman: Deutsche Anführungszeichen im Literaturverzeichnis (thx @ Markus Brenneis)
%   * ngerman: Deutsche Anführungszeichen im Literaturverzeichnis (thx @ Markus Brenneis)
% 16.10.2018 (Matthias Liebeck): Zwei fixes an \sectionforcestartright (thx @ Markus Brenneis)
%%%%%%%%%%%%%%%%%%%%%%%%%%%%%%%%%%%%%%%%%%%%%%%%%%%%%%%%%%%%%%%%%%%%%%%%
%%%% BEGINN EINSTELLUNG FUER DIE ARBEIT. UNBEDINGT ERFORDERLICH! %%%%%%%
%%%%%%%%%%%%%%%%%%%%%%%%%%%%%%%%%%%%%%%%%%%%%%%%%%%%%%%%%%%%%%%%%%%%%%%%
% Geben Sie Ihren Namen hier an:
\raggedbottom
\newcommand{\bearbeiter}{Hanna Pankova}

% Geben Sie hier den Titel Ihrer Arbeit an:
\newcommand{\titel}{AI-based fluorescent labeling for cell line development}

% Geben Sie das Datum des Beginns und Ende der Bachelorarbeit ein:
\newcommand{\beginndatum}{01. April 2022}
\newcommand{\abgabedatum}{29. August 2022}

% Geben Sie die Namen des Erst- und Zweitgutachters an:
\newcommand{\erstgutachter}{Prof. Dr. Markus Kollmann}
\newcommand{\zweitgutachter}{Dr. Wolfgang Halter}

% Falls Sie die Arbeit zweiseitig ausdrucken wollen,
% benutzen Sie die folgende Zeile mit
% \AN fuer zweiseitigen Druck
% \AUS fuer einseitigen Druck
\newcommand{\zweiseitig}{\AN}
% true fuer biber, false fuer klassischen Zitierstil
%\newcommand{\biber}{false}
\newcommand{\biber}{true}

% Falls Sections immer rechts beginnen sollen. Gerade für Masterarbeiten
% interessant. Bei kurzen Bachelorarbeiten eher weniger zu verwenden.
\newcommand{\sectionforcestartright}{false}
%\newcommand{\sectionforcestartright}{true}

% Falls die Arbeit in englischer Sprache verfasst
% werden soll, dann benutzen Sie die folgende Zeile mit
% englisch fuer englische Sprache
% deutsch fuer deutsche Sprache
\newcommand{\sprache}{englisch}

% Hier wird eingestellt, ob es sich bei der Arbeit um eine Bachelor-
% oder Masterarbeit handelt (unpassendes auskommentieren!):
\newcommand{\arbeit}{Master thesis}
%~ \newcommand{\arbeit}{Masterarbeit}


%%%%%%%%%%%%%%%%%%%%%%%%%%%%%%%%%%%%%%%%%%%%%%%%%%%%%%%%%%%%%%%%%%%%%%%%
%%%% ENDE EINSTELLUNGEN %%%%%%%%%%%%%%%%%%%%%%%%%%%%%%%%%%%%%%%%%%%%%%%%
%%%%%%%%%%%%%%%%%%%%%%%%%%%%%%%%%%%%%%%%%%%%%%%%%%%%%%%%%%%%%%%%%%%%%%%%

% Die folgende Zeile NICHT EDITIEREN oder loeschen


%%%%%%%%%%%%%%%%%%%%%%%%%%%%%%%%%%%%%%%%%%%%%%%%%%%%%%%%%%%
% Obere Titelmakros. Editieren Sie diese Datei nur, wenn
% Sie sich ABSOLUT sicher sind, was Sie da tun!!!
% (Z.B. zum Abaendern der BA-Vorlage in eine MA-Vorlage)
% Uni Duesseldorf
% Lehrstuhl fuer Datenbanken und Informationssysteme
% Version 2.2 - 2.3.2010
%%%%%%%%%%%%%%%%%%%%%%%%%%%%%%%%%%%%%%%%%%%%%%%%%%%%%%%%%%%
\newcommand{\AN}{twoside}
\newcommand{\AUS}{}


%\newcommand{\englisch}{}
%\newcommand{\deutsch}{\usepackage[german]{babel}}

%% Die folgenden auskommentierten Optionen dienen der automatischen
%% Erkennung des Latex-Kompilers und dem Setzen der davon abhängigen
%% Einstellungen. Bei Problem z.B. mit dem Einbinden von verschiedenen
%% Grafiktypen bei Verwendung von PdfLatex oder Latex, einfach die
%% verschiedenen \usepackage(s) ausprobieren. (Mit diesen Einstellungen
%% funktionierte diese Vorlage bei der Verwenundg von latex.exe als
%% Kompiler bei den meisten Studierenden.)

%\newif\ifpdf \ifx\pdfoutput\undefined
%\pdffalse % we are not running pdflatex
%\else
%\pdfoutput=1 % we are running pdflatex
%\pdfcompresslevel=9 % compression level for text and image;
%\pdftrue \fi

\documentclass[11pt,a4paper, \zweiseitig]{article}
\usepackage{ifthen}


%\usepackage[iso]{umlaute}
\usepackage[utf8]{inputenc}
\usepackage{palatino} % palatino Schriftart
%\usepackage{makeidx} % um ein Index zu erstellen
\usepackage[nottoc]{tocbibind}
\usepackage[T1]{fontenc} %fuer richtige Trennung bei Umlauten
\usepackage{fancybox} % fuer die Rahmen
\usepackage{shortvrb}
%\usepackage{slashbox}
\usepackage{diagbox}
\usepackage{url}
\usepackage{adjustbox}
\usepackage{xcolor}
\usepackage{multicol}
\usepackage{graphicx}
\usepackage{caption}
\usepackage{amsthm}
\usepackage{amssymb}
\usepackage{titlesec}
\usepackage{mathrsfs}
\usepackage{dsfont}
\usepackage{amsmath}
\usepackage{tabularx}
\usepackage{float}
\usepackage{listings}
\usepackage{color}
\usepackage{algorithm}
\usepackage{algpseudocode}
\usepackage{subcaption}
\usepackage{listings}
\usepackage{physics}

\titlespacing*{\section}{0pt}{5.5ex plus 1ex minus .2ex}{4.3ex plus .2ex}
\titlespacing*{\subsection}{0pt}{5.5ex plus 1ex minus .2ex}{4.3ex plus .2ex}

\definecolor{dkgreen}{rgb}{0,0.6,0}
\definecolor{gray}{rgb}{0.5,0.5,0.5}
\definecolor{mauve}{rgb}{0.58,0,0.82}

\lstset{frame=tb,
  language=Python,
  aboveskip=3mm,
  belowskip=3mm,
  showstringspaces=false,
  columns=flexible,
  basicstyle={\small\ttfamily},
  numbers=none,
  numberstyle=\tiny\color{gray},
  keywordstyle=\color{blue},
  commentstyle=\color{dkgreen},
  stringstyle=\color{mauve},
  breaklines=true,
  breakatwhitespace=true,
  tabsize=3
}
\newcolumntype{Y}{>{\centering\arraybackslash}X}
\theoremstyle{definition}
\newtheorem{definition}{Definition}[section]
\usepackage[colorlinks,citecolor=blue,linkcolor=black]{hyperref} %anklickbares Inhaltsverzeichnis

\ifthenelse{\boolean{\biber}}{
  % only needed for biber
  \usepackage[style=authoryear,natbib=true,backend=biber,mincitenames=1,maxcitenames=2,maxbibnames=99,uniquelist=false,dashed=false]{biblatex}

  % https://tex.stackexchange.com/a/334703/8850
  \AtEveryBibitem{%
    \clearfield{issn}
    \clearfield{isbn}
    \clearfield{doi}
    \clearfield{location}
    \clearlist{location}
    \clearlist{address}

    \ifentrytype{online}{}{% Remove url except for @online
      \clearfield{url}
    }
  }
}
{}%no else

% Falls es bei \citet ein Komma zwischen Name und Jahr gibt:
% https://tex.stackexchange.com/questions/312539/unwanted-comma-between-author-and-year-using-citet-command
% (thx @ Markus Brenneis)
%\DeclareDelimFormat[cbx@textcite]{nameyeardelim}{\addspace}



\ifthenelse{\equal{\sprache}{deutsch}}{
  \usepackage[ngerman]{babel}
  % Bibtex u.a -> et al.
  \ifthenelse{\boolean{\biber}}{
    \DefineBibliographyStrings{ngerman}{
      andothers = {{et\,al\adddot}},
    }
    \newcommand{\references}{Literatur}
  }
  {} % do nothing when not using biber
  \usepackage[autostyle, german=quotes]{csquotes} % Deutsche Anführungszeichen im Literaturverzeichnis (thx @ Markus Brenneis)

}{ \newcommand{\references}{References}}

\usepackage{a4wide} % ganze A4 Weite verwenden



%\ifpdf
%\usepackage[pdftex,xdvi]{graphicx}
%\usepackage{thumbpdf} %thumbs fuer Pdf
%\usepackage[pdfstartview=FitV]{hyperref} %anklickbares Inhaltsverzeichnis
%\else
%\usepackage[dvips,xdvi]{graphicx}
\usepackage{graphicx}

%\fi

\newcommand{\redt}[1] {
  \textcolor{red}{#1}}

\newcommand{\oranget}[1] {
  \textcolor{orange}{#1}}

\newcommand{\purplet}[1] {
  \textcolor{purple}{#1}}

%%%%%%%%%%%%%%%%%%%%%%% Massangaben fuer die Arbeit %%%%%%%%%%%%%%%
\setlength{\textwidth}{15cm}

\setlength{\oddsidemargin}{35mm}
\setlength{\evensidemargin}{25mm}

\addtolength{\oddsidemargin}{-1in}
\addtolength{\evensidemargin}{-1in}

\ifthenelse{\boolean{\biber}}{\addbibresource{references.bib}}{}

%\makeindex
\begin{document}
%setcounter{secnumdepth}{4} %Nummerieren bis in die 4. Ebene
%\setcounter{tocdepth}{4} %Inhaltsverzeichnis bis zur 4. Ebene

\pagestyle{headings}

\sloppy % LaTeX ist dann nicht so streng mit der Silbentrennung
%~ \MakeShortVerb{\§}

\parindent0mm
\parskip0.5em


{
\textwidth170mm
\oddsidemargin30mm
\evensidemargin30mm
\addtolength{\oddsidemargin}{-1in}
\addtolength{\evensidemargin}{-1in}

\parskip0pt plus2pt

% Die Raender muessen eventuell fuer jeden Drucker individuell eingestellt
% werden. Dazu sind die Werte fuer die Abstaende `\oben' und `\links' zu
% aendern, die von mir auf jeweils 0mm eingestellt wurden.

%\newlength{\links} \setlength{\links}{10mm}  % hier abzuaendern
%\addtolength{\oddsidemargin}{\links}
%\addtolength{\evensidemargin}{\links}

\begin{titlepage}
\vspace*{-1.5cm}
\raisebox{17mm}{
    \begin{minipage}[t]{70mm}
        \begin{center}
            %\selectlanguage{german}
            {\Large INSTITUTE OF COMPUTER SCIENCES\\}
            {\normalsize
                Master in Artificial Intelligence and Data Science\\
            }
            \vspace{3mm}
            {\small Universitätsstr. 1 \hspace{5ex} D--40225 Düsseldorf\\}
        \end{center}
    \end{minipage}
}
\hfill
\raisebox{7mm}{
    \includegraphics[width=130pt]{bilder/HHU_Logo}}
\vspace{14em}

% Titel
\begin{center}
    \baselineskip=55pt
    \textbf{\huge \titel}
    \baselineskip=0 pt
\end{center}

%\vspace{7em}

\vfill

% Autor
\begin{center}
    \textbf{\Large
        \bearbeiter
    }
\end{center}

\vspace{35mm}

% Prüfungsordnungs-Angaben
\begin{center}
%\selectlanguage{german}

%%%%%%%%%%%%%%%%%%%%%%%%%%%%%%%%%%%%%%%%%%%%%%%%%%%%%%%%%%%%%%%%%%%%%%%%%
% Ja, richtig, hier kann die BA-Vorlage zur MA-Vorlage gemacht werden...
% (nicht mehr nötig!)
%%%%%%%%%%%%%%%%%%%%%%%%%%%%%%%%%%%%%%%%%%%%%%%%%%%%%%%%%%%%%%%%%%%%%%%%%
{\Large \arbeit}

\vspace{2em}
\ifthenelse{\equal{\sprache}{deutsch}}{
    \begin{tabular}[t]{ll}
    Beginn der Arbeit:& \beginndatum \\
    Abgabe der Arbeit:& \abgabedatum \\
    Gutachter:         & \erstgutachter \\
                        & \zweitgutachter \\ 
                        & \dreigutachter \\
}{
    \begin{tabular}[t]{ll}
    Date of issue:& \beginndatum \\
    Date of submission:& \abgabedatum \\
    Reviewers:         & \erstgutachter \\
                        & \zweitgutachter \\ 
                        & \dreigutachter \\
}
\end{tabular}
\end{center}

\end{titlepage}

}

%%%%%%%%%%%%%%%%%%%%%%%%%%%%%%%%%%%%%%%%%%%%%%%%%%%%%%%%%%%%%%%%%%%%%
\clearpage
\begin{titlepage}
    ~                % eine leere Seite hinter dem Deckblatt
\end{titlepage}
%%%%%%%%%%%%%%%%%%%%%%%%%%%%%%%%%%%%%%%%%%%%%%%%%%%%%%%%%%%%%%%%%%%%%
\clearpage
\begin{titlepage}
    \vspace*{\fill}

    \section*{Erklärung}

    %%%%%%%%%%%%%%%%%%%%%%%%%%%%%%%%%%%%%%%%%%%%%%%%%%%%%%%%%%%
    % Und hier ebenfalls ggf. BA durch MA ersetzen...
    % (Auch nicht mehr nötig!)
    %%%%%%%%%%%%%%%%%%%%%%%%%%%%%%%%%%%%%%%%%%%%%%%%%%%%%%%%%%%

    Hiermit versichere ich, dass ich diese \arbeit{}
    selbstständig verfasst habe. Ich habe dazu keine anderen als die
    angegebenen Quellen und Hilfsmittel verwendet.

    \vspace{25 mm}

    \begin{tabular}{lc}
        Düsseldorf, den \abgabedatum \hspace*{2cm} & \underline{\hspace{6cm}} \\
                                                   & \bearbeiter
    \end{tabular}

    \vspace*{\fill}
\end{titlepage}

%%%%%%%%%%%%%%%%%%%%%%%%%%%%%%%%%%%%%%%%%%%%%%%%%%%%%%%%%%%%%%%%%%%%%
% Leerseite bei zweiseitigem Druck
%%%%%%%%%%%%%%%%%%%%%%%%%%%%%%%%%%%%%%%%%%%%%%%%%%%%%%%%%%%%%%%%%%%%%

\ifthenelse{\equal{\zweiseitig}{twoside}}{\clearpage\begin{titlepage}
        ~\end{titlepage}}{}

%%%%%%%%%%%%%%%%%%%%%%%%%%%%%%%%%%%%%%%%%%%%%%%%%%%%%%%%%%%%%%%%%%%%%
\clearpage
\begin{titlepage}

    %%% Die folgende Zeile nicht ändern!
\section*{\ifthenelse{\equal{\sprache}{deutsch}}{Abstract}{Abstract}}
%%% Zusammenfassung:
Cell line development is an expensive and time-consuming process, however that is the most modern approach for producing the proteins needed in various pharmaceuticals.


    %%%%%%%%%%%%%%%%%%%%%%%%%%%%%%%%%%%%%%%%%%%%%%%%
    % Untere Titelmakros. Editieren Sie diese Datei nur, wenn Sie sich
    % ABSOLUT sicher sind, was Sie da tun!!!
    %%%%%%%%%%%%%%%%%%%%%%%%%%%%%%%%%%%%%%%%%%%%%%%
    \vspace*{\fill}
\end{titlepage}

%%%%%%%%%%%%%%%%%%%%%%%%%%%%%%%%%%%%%%%%%%%%%%%%%%%%%%%%%%%%%%%%%%%%%
% Leerseite bei zweiseitigem Druck
%%%%%%%%%%%%%%%%%%%%%%%%%%%%%%%%%%%%%%%%%%%%%%%%%%%%%%%%%%%%%%%%%%%%%
\ifthenelse{\equal{\zweiseitig}{twoside}}
{\clearpage\begin{titlepage}~\end{titlepage}}{}
%%%%%%%%%%%%%%%%%%%%%%%%%%%%%%%%%%%%%%%%%%%%%%%%%%%%%%%%%%%%%%%%%%%%%
\clearpage \setcounter{page}{1}
\pagenumbering{roman}
\setcounter{tocdepth}{4}
\tableofcontents

%\enlargethispage{\baselineskip}
\clearpage
%%%%%%%%%%%%%%%%%%%%%%%%%%%%%%%%%%%%%%%%%%%%%%%%%%%%%%%%%%%%%%%%%%%%%
% Leere Seite, falls Inhaltsverzeichnis mit ungerader Seitenzahl und
% doppelseitiger Druck
%%%%%%%%%%%%%%%%%%%%%%%%%%%%%%%%%%%%%%%%%%%%%%%%%%%%%%%%%%%%%%%%%%%%%
\ifthenelse{ \( \equal{\zweiseitig}{twoside} \and \not \isodd{\value{page}} \)}
{\pagebreak \thispagestyle{empty} \cleardoublepage}{\clearpage}


% Kapitel soll bei doppelseitigem Druck immer auf der rechten (ungeraden) Seite anfangen (thx @ Philipp Grawe)
% https://tex.stackexchange.com/a/223387
\ifthenelse{\boolean{\sectionforcestartright}}
{\let\oldsection\section % Store \section in \oldsection
    \renewcommand{\section}{\cleardoublepage\oldsection}}
{}
\pagenumbering{arabic}
\setcounter{page}{1}

%%%%%%%%%%%%%%%%%%%%%%%%%%%%%%%%%%%%%%%%%%%%%%%%%%%%%%%%%%%%%%%%%%%%%%%%
%%%% BEGINN TEXTTEIL %%%%%%%%%%%%%%%%%%%%%%%%%%%%%%%%%%%%%%%%%%%%%%%%%%%
%%%%%%%%%%%%%%%%%%%%%%%%%%%%%%%%%%%%%%%%%%%%%%%%%%%%%%%%%%%%%%%%%%%%%%%%

%%%%%%%%%%%%%%%%%%%%%%%%%%%%%%%%%%%%%%%%%%%%%%%%%%%%%%%%%%%%%%%%%%%%%%%%
% Text entweder direkt hier hinein schreiben oder, im Sinne der
% besseren Uebersichtlich- und Bearbeitbarkeit mittels \input die
% einzelnen Textteile hier einbinden.
%%%%%%%%%%%%%%%%%%%%%%%%%%%%%%%%%%%%%%%%%%%%%%%%%%%%%%%%%%%%%%%%%%%%%%%%

\section{Introduction}
    \subsection{Motivation}
    Nowadays recombinant proteins are widely used in biomedical research and production of medicines that are used in the variety of therapeutic neeeds like vaccines and antibodies. Therefore there is currently a great need for high-volume and high-quality recombinant protein production. That is why optimization and improvement of cell line development (CLD) as a process that is used for production of recombinant proteins is extremely important.

Clone screening is a step of the CLD process where cells are analyzed for further selection of the most stable and productive clones. Fluorescence microscopy provides data about the cell structure that allows performing better clone selection, however it is not only expensive and time-consuming, but also toxic for the cells. Therefore automating fluorescence microscopy for clone selection via convolutional neural networks \textit{in silico} significantly simplifies the existing procedure of clone selection, reduces phototoxicity time and expenses needed for the analysis.

The goal of this thesis is to provide a proof of concept on whether an \textit{in silico} approach for fluorescent labeling can substitute manual cell staining and provide all the needed information that would be used for further clone screening and selection. That is why this research is aimed towards the specific needs, pipelines and data used at Merck KgaA. In this research four UNet models (for four target proteins highlighting different cell organelles) were developed for automating fluorescence cell staining based on DIC microscopy imaging of CHO cells: nuclei, endoplasmic reticulum, [[green fluorescent protein]] and Golgi apparatus. Another important goal of this research that differentiates it from the similar studies like [TODO cite LaChance 2020 and cite Christiansen 2018] is to not only provide deep learning models for the fluorescence predicions but also study their reliability and be able to detect drift during image acquisition that can happen quite easily due to the sensitivity of the microsope settings and well as the cell phenotypes, scaling and fixation procedures.

This thesis is laid out as follows: Section 2 provides a review of the biological concepts needed to understanding the application of this research and well as the machine and deep learning concepts used for data analysis; Section 3 provides an overview of the implementation and the results of the experimental \textit{in silico} fluorescence predictions; Section 4 reviews stability of the deep learning models developed in the previous section and provides valuable insights on the information from their embeddings; Section 5 reviews the practical tools used for the development at Merck KgaA and Section 6 explores possible future research questions that arose from the current analysis and provides concluding remarks.

    \pagebreak
    %\subsection{Notation}
    %\nomenclature{$p_{data}$}{Data generating distribution}
\nomenclature{$\bm{X}_{train}$}{A set of training examples}
\nomenclature{$\bm{x^{(i)}}$}{The i-th input image (sample) from a dataset}
\nomenclature{$\bm{y^{(i)}}$}{The target image associated with the i-th input sample from a dataset}
\nomenclature{$A_{i, j}$}{An element on the i-th row and j-th column of a matrix $A$}
\nomenclature{$\mathbb{R}$}{A set of real numbers}

\printnomenclature
\pagebreak
\section{Domain knowledge}
% taken from here https://pubs.acs.org/doi/full/10.1021/acsphotonics.2c00599?casa_token=_MabQ6pGe48AAAAA%3A3cKiQjee69lw88NnkGzeH3OiTfHFd71Z4NjOJWBpdIUMNYMERNJ6mu9UpaTOYhZT7K8nlmxvJf7EqLHD

    The \textit{in silico} fluorescence labeling approach has proven to be very promising as a substitute to the manual cell staining processes [TODO cite all the relevant references]. For example, the research of [TODO cite Christiansen 2018] did not only prove successful prediction of different cell stains with a variety of modalities and cell types, but it had also successfully determined cell viability. Nevertheless, the study is limited mainly to transmitted light (TL) z-stack imaging. This refers to the networks input being comprised of 3D images, which is not the case in this work. [cite Ounkomol 2018] too shows succeful predictions of several organelles in bright-field TL 3D images using 3D convolutional neural networks. However, switching to 2D data did not yield adequate results for them. Other, newer studies like [cite Ugawa 2021] provide an application of label-free fluorescence predicting already at the sorting stage, when a high-throughput system sorts cells individually. However, only a single-pixel detector is used by this study, meaning that it captures a wave rather than an image. Nonetheless one can recover an image with heavy computations if needed [cite Sadao Ota 2018].
    %https://www.science.org/doi/10.1126/science.aan0096
    
    % [Boustany 2010 https://www.ncbi.nlm.nih.gov/pmc/articles/PMC3357207/]
    There are two very promising studies by [cite Cheng 2021] and [cite LaChance 2020]. Even though the former manages to reach a state-of-the art performance on label-free fluorescence reconstruction, it uses reflectance images from oblique dark-field illumination as the input, which is a more specific cell imaging approach. Still, this input provides higher structural contrast in comparison to any transmission technique [cite Boustany 2010]. The latter study uses an easier imaging technique (DIC imaging) as an input, which shows great results even with low-resolution data. Both of these studies provide results based not only on training metrics, but also on performance of the models for metrics used in the downstream tasks. This is very important in the label-free fluorescence labeling research and was not present in papers before LaChance. In the thesis at hand, many methods from the LaChance paper were used as both the data and the processes in the project pipeline of Merck KgaA align very well with the study conducted in that paper.
    
    All of the studies mentioned above, as well as this work rely on the premise that the input imaging type (here DIC) contains enough information to predict the fluorescence signal from it. This is a reasonable assumption because DIC, as well as bright-field and phase contrast imaging, are very often used for determining cell morphology [TODO cite Kasprowicz 2017].

    This chapter provides a brief overview of the biological background needed to understand the process of cell line development (CLD) and the role of fluorescent \textit{in silico} labeling of DIC cell images within. It also covers the fundamentals of deep and machine learning techniques used here including clustering and dimensionality reduction approaches. At the end of the chapter, a brief summary of the microscopy image acquisition process used in the research is given.

    \subsection{Biology}
        \subsubsection{Cell line development process}
        The cell line development (CLD) is a process of generating single cell-derived clones that produce high and consistent levels of target therapeutic protein (\cite{lonza}). Therapeutic proteins in this case are so-called recombinant proteins and they are widely used in biomedical research, the production of medication and for various therapeutic needs such as, for example, vaccines and monoclonal antibodies (mAbs) (\cite{Ohtake_2013}, \cite{Jefferis_2017}, \cite{Funaro_1996}). A recombinant protein, as defined by \cite{Barbeau_2018}, is a modified or manipulated protein encoded by a recombinant DNA. Recombinant DNA in turn consists of a plasmid, where the genes of the target protein of interest are cloned downstream of a promoter region. As soon as this plasmid is transfected to a host cell (for example some mammalian cells that are able to produce the protein), the host will start to express this protein of interest. Today there is a great need for the production of high volumes of good quality recombinant proteins, both in industrial as well as research contexts (\cite{Tihanyi_2020}). For this reason the goal of many research projects in recombinant protein production is to improve expression efficiency and create high-throughput systems to improve the CLD processes (\cite{Tihanyi_2020}).

% TODO \cite[see]{IWNLP} \supercite{iqbal2007underwater}

% TODO [cite Beckman] was for "remain the most popular choice"

One of the most popular host cells used in CLD and in this thesis specifically are chinese hamster ovary (CHO) cells (\cite{Castan_2018}). Although different cells can be used as hosts, such as bacterial, plant-based or yeast cells, mammalian cells remain the most popular choice. The reason behind this popularity resides in the fact that they can produce a diverse range of correctly folded proteins and most importantly they have high protein production rates. The productivity rate is measured in titre of produced protein, and CHO cells can reach 0.1 - 1 g/L in batch and 1 - 10 g/L in fed-batch cultures (\cite{Tihanyi_2020}). Mostly all of the mAbs are produced using CHO cells (\cite{Lalonde_2017}). Pharmaceutical companies very often try to use the same host cell line for all their productions because already checked and qualified cells simplify the road to the clinic (\cite{Tihanyi_2020}). Since this research is dedicated to CHO cell line as well, it has a wide applicability.

However, there is a downside to using CHO as host cells ---- these rapidly growing immortal cells are also genomically unstable and extremely heterogeneous which usually leads to the main issue: production instability. Instability here means that the cell might die or do not produce target protein. The problem of choosing stable and high-production clones that simultaneously will be able to express protein qualitatively and quantifiably over time is essentially the main goal of current research. The challenge in manufacturing here is the time and the cost of production. Currently, a lot of research attention is dedicated to the reduction of both factors, as well as the development of techniques of high-throughput clone screening and characterization (\cite{Tihanyi_2020}). The latter is of interest for this thesis. With great amounts of data collected over time and the development of computational modelling and statistical analysis it is now possible to carry out the analysis \textit{in silico}, meaning computationally without interfering with the cells instead of the usual \textit{in vitro} analysis (\cite{Christiansen_2018}), which will be also shown in this research.

\paragraph{CLD steps}
\label{section:cld-steps}
\begin{figure}[H]
	\begin{center}
		\includegraphics[width=0.8\linewidth]{bilder/CLD.png}
		\caption[CLD process steps]%
		{CLD process steps. Usual times needed for this process: from transfection to characterization --- 5 months, stability --- 3 months. Minipool selection step that is optimized here is taking up to 5 weeks.}\label{fig:cls-steps}
	\end{center}
\end{figure}

The first step of CLD is called transfection --- the introduction of the gene of interest (abbreviated as GOI or can be called a DNA vector or, alternatively, an expression vector) into CHO cells. There are two main challenges with this step: firstly, transfection mostly results in a vector being inserted into a random site within the host cell genome and secondly, it generally has low efficiency of integration  (\cite{Tihanyi_2020}). It is important to transfect a GOI into the optimal site of the genome to secure high protein expression over time during protein production. Practically however, GOI is transfected into a random location of the genome. In cases where the gene was transfected into an inactive site of genome (essentially the majority of genome is transcriptionally inactive), the cell will likely be unable to express the gene (\cite{Castan_2018}, \cite{Hong_2018}).

%TODO [a better reference needed Shin 2020]
The second step of the process is the selection of cell minipools that have successful and stable gene integrations for further expansion and cloning. The reason for not all of them being suitable is that during the transfection step, only 80\% of the cells will receive a GOI vector (\cite{Castan_2018}). Only a small percentage of these cells actually integrate a vector into the genome and, as mentioned above, only a fraction of those are able to express the protein in a stable fashion (\cite{Shin_2020}). After the best minipools are selected, they will be expanded, which means that cell population is serially passaged to a larger population with the bigger number of cells.

The third step in CLD is to clone the cells. The selected stable pools of cells are phenotypically and genetically diverse. This means that they have different growth rates, metabolic profiles, and so forth. This is not ideal for industrial production - all the cells used for protein production should be derived from the same clone (\cite{ema_2020}). 

Once the cells are cloned, phenotypical and genetical heterogeneity is reduced, the next step is to characterize the cells for their expression of the GOI. One has to estimate the clones' productivity and stability. Such observations may take up to 90 days (usually stability measurements are made on the $30^{\text{th}}$, $60^{\text{th}}$ and $90^{\text{th}}$ days). If the clones remain stable after this time and are able to express enough of the protein, then they are suitable for further production. However, this last step is very time-consuming and requires maintenance for feeding and analysing the cells. Predicting productivity and stability of the cells in earlier stages would reduce this time significantly or even allow to avoid this process entirely.
        \subsubsection{Project specifications of cell line development for Merck KgaA}
        There are many different proteins that can be produced using such technologies, for example, vaccines, hormones, sugars etc., however this research is dedicated to the production of monoclonal antibodies (mAbs). 

CHOZN® Platform is a currently widely used product of Merk KgaA. CHOZN is a CHO mammalian cell expression system for fast and easy selection and growth of clones producing high levels of recombinant proteins [cite tech-bulletin]. The processes of developing expression systems on this platform correspond to the general CLD process described in the previous subsection [put subsection number]. The scope of the project is to simplify the labour-intensive and time-consuming process of stability measurement of the expression system by inducing predictions of productivity and stability rates during early steps in the CLD process. 

After the transfection step there are several quantities that are measured in minipools in order to select the best ones. For example, cell size, its complexity, cell surface protein expression, endoplasmic reticulum (ER) mass, mitochondria mass, etc. For qualitative and quantitative characterization of cells, fluorescent labeling is used. It is a process of covalently binding fluorescent dyes to biomolecules such as nucleic acids or proteins, so that they can be visualized via fluorescence imaging [cite https://www.nature.com/subjects/fluorescent-labelling]. A fluorophore is a chemical compound that can reemit light at a certain wavelength.These compounds are a critical tool in biology because they allow experimentators to capture particular components of a given cell in detail [cite O'reilly life sciences p113].

Unfortunately, fluorescence labeling is expensive, time-consuming and may kill the cell due to its phototoxicity [cite Fried et al., 1982; Patil et al., 2018; Progatzky et al., 2013)]. Additionally, Yeo et al. [cite Tihanyi] found out that different selection markers affect the production stability of CHO cells. Other negative aspects of manual staining appoach are: there is a limited number of available fluorescent channels in microscopes; some fluorophores have a spectral overlap, hence there is a limited number of detectable markers [cite Perfetto et al., 2004]; such labels can be inconsistent [cite Burry, 2011; Weigert et al., 1970), and depend a lot on reagent quality and require many hours of lab work. Toxicity, for instance, is a very dangerous factor, especially for medicine production as it may even affect the final product. Therefore there exists a need for an approach of \textit{in silico} flurescent labeling - computationally and without affecting the cell. 

For \textit{in silico} labeling, the input data is a differential interference contrast (DIC) microscopy. This is an optical microscopy technique used to enhance the contrast in unstained, transparent samples [cite wikipedia?]. This is a much cheaper image acquision technique than a staining process, and it has much less variability as well (for example, no dependency on the dye or antibody quality). The research is dedicated to predicting fluorescence signal from the DIC imaging directly without the need of actual cell staining. The measurements needed for selection of minipools can be calculated as usual, but using the predicted images instead.

    \subsection{Deep learning and machine learning basics}
        Introduction of the notaiton for the dataset, parameters, predictions.
        \subsubsection{Neural networks}
            \begin{definition}[Image dataset]
	An image dataset in the scope of this thesis constists of input DIC images $X$ and target fluorescence images $Y$. Combined, couples from each form (X and Y) construct a dataset:
	\begin{equation}
		D = (X, Y) = \{(x^{(1)}, y^{(1)}), \dots, (x^{(N)}, y^{(N)})\}
	\end{equation}

	where both $x^{(i)}$ and $y^{(i)} \in \mathbb{R}^{W \times H}$ are single images, $N$ is the size of the dataset. Generally input data has a shape of $(N, C, H, W)$, in this work $C = 1$.
\end{definition}

\begin{definition}[Model]
	A model is a function with learnable parameters $\theta = (\theta_1, ..., \theta_K)$ where $\theta_i \in \mathbb{R}$ for $i \in {0, ..., K}$ which approximates the mapping of initial data $X$ to target data $Y$.
	\begin{equation}
		M(X,\theta) = Y^\prime \approx Y 
	\end{equation}
\end{definition}

\begin{definition}[Loss function]
	A loss function is a function $L(y, M(x, \theta))$ of model's parameters $\theta$, that for $(x^{(i)}, y^{(i)}) \in D$ outputs a scalar value measuring the difference between ground truth $y$ and prediction $M(x, \theta)$. A training objective is then defined as an average over the loss of each training sample:
	\begin{equation}
		J(\theta) = \mathbb{E}_{(x, y)\sim p_{data}} L(y, M(x, \theta))
	\end{equation}
	where $p_{data}$ denotes an empirical distribution of the training data.
\end{definition}

\begin{definition}[Binary-cross entropy loss]
	Let $y \in \mathbb{R}^{W \times H}$ be a ground truth image and $y^\prime \in \mathbb{R}^{W \times H}$ be a prediction. Binary-cross entropy loss is defined as:
	\begin{equation}
		L(y, y^\prime) = - \frac{1}{N^2}\sum_{i=1}^{H} \sum_{j=1}^{W} y_{i,j} \cdot \log(y_{i, j}^\prime) +  (1 - y_{i, j}) \cdot \log(1 - y_{i, j}^\prime) 
	\end{equation}
\end{definition}

\begin{definition}[MSE (mean squared error) loss]
	Let $y \in \mathbb{R}^{W \times H}$ be the ground truth and $y^\prime \in \mathbb{R}^{W \times H}$ be the predicted images. The MSE loss is defined as:
	\begin{equation}
		L(y, y^\prime) = \sum_{i=1}^{H} \sum_{j=1}^{W} (y_{i, j} - y_{i, j}^\prime)^2
	\end{equation}
\end{definition}

\begin{definition}[PCC (Pearson correlation coefficient) loss]
	\label{def:pcc-loss}
	Let $y \in \mathbb{R}^{WH}$ be a flattened ground truth and $y^\prime \in \mathbb{R}^{WH}$ be a flattened predicted image. The PCC loss is defined as:
	\begin{align}
		PCC(y, y^\prime) &= \frac{\sum_{i=1}^{{WH}}{(y_i - \bar{y})(y_i^\prime - \bar{y}^\prime)}}{\sqrt{\sum_{i=1}^{{WH}^2}{(y_i - \bar{y})^2(y_i^\prime - \bar{y}^\prime)^2}}}  \\
		L(y, y^\prime) &= \frac{1 - PCC(y, y^\prime)}{2}
	\end{align}
	where $\bar{y}$, $\bar{y}^\prime$ are means of the ground truth and predicted images respectively.
	
	There is an important distinction to be made here: firstly, Pearson correlation coefficient (PCC further) in a measure of similarity between two data sequences (matrices in this case), with values between $-1$ and $1$, with $1$ being a positive correlation, secondly, PCC loss is a measure of dissimilarity between two matrices, with values between $0$ and $1$, with $0$ meaning that matrices are the same.

	This loss is widely used in cell biology for comparison of co-localization between the proteins (\cite{Lachance_2020}). PCC is also popular in computer vision where it is utilized for the determination of image similarity in terms of spatial-intensity (\cite{Lachance_2020}).
\end{definition}

\begin{definition}[Optimization]
	Optimization is a process of updating the parameters $\theta$ of the model $M(X, \theta)$ to minimize the loss function $L(y, M(x, \theta))$.
\end{definition}

With a maximum likelihood esimation, we get:
\begin{equation}
	\theta_{MLE} = \argmax\limits_{\theta} \sum_{i=1}^{N} \log{p_{\text{model}}(x^{(i)}, y^{(i)}, \theta)}
\end{equation}

After maximizing the sum and taking a gradient one gets:
\begin{equation}
	\nabla_{\theta} J(\theta) = \mathbb{E}_{x, y \sim p_{data}} \nabla_{\theta} \log{p_{\text{model}}(x, y, \theta)}
\end{equation}

The exact gradient on a discretized data-generating distribution is then:
\begin{equation}
	g = \nabla_{\theta} J^*(\theta) = \sum_{x} \sum_{y}{p_{\text{data}}(x, y) \nabla_{\theta} L(y, M(x, \theta))}
\end{equation}

Here one can obtain an unbiased estimator of a true gradient on a mini-batch of i.i.d. samples $\{x^{(i)}, ..., x^{(m)}\}$	

\begin{equation}
	\hat{g} = \frac{1}{m} \nabla_\theta \sum_{i} L(y^{(i)}, M(x^{(i)}, \theta))
\end{equation}

\begin{definition}[Stochastic gradient descent]
	Stochastic gradient descent is an optimization algorithm where the parameters $\theta$ are iteratively updated every mini-batch of data by the following rule:
	\begin{equation}
		\theta_{k+1} = \theta_k - \alpha \frac{1}{m} \nabla_\theta \sum_{i} L(y^{(i)}, M(x^{(i)}, \theta))
	\end{equation}
	where $\alpha$ is a tuneable parameter called \textit {learning rate}.
\end{definition}

\begin{definition}[Adadelta optimizer]
	An Adadelta optimizer is a more sophisticated optimization technique, that follows algorithm \ref{alg:adadelta} for the parameter update.
	\begin{algorithm}[H]
		\caption{Adadelta optimization}\label{alg:adadelta}
		\item 1. $E[g]^2_0 = 0$ and $E[\Delta \theta^2]_0 = 0$
		In order to update the parameters one needs to:
		\item 2. Compute gradient: $g_t$
		\item 3. Accumulate gradient: $E[g]^2_t = \rho E[g]^2_{t - 1} + (1 - \rho)g_t^2$
		\item 4. Compute update: $\Delta \theta_t = \frac{\text{RMS}[\Delta \theta]_{t-1}}{\text{RMS}[g]_t} \hat{g_t}$
		\item 5. Accumulate updates: $E[\Delta \theta^2]_t = \rho E[\Delta \theta^2]_{t-1} + (1 - \rho) \Delta \theta^2_t$
		\item 6. Apply update: $\theta_{t+1} = \theta_t + \Delta \theta_t$ \\
		RMS here is the root mean square all initial hyperparametes are take from the original study(\cite{Zeiler_2012}).
	\end{algorithm}
\end{definition}

\begin{definition}[Adam optimizer]
	An Adam optimizer is another stohastic optimization technique, that has the following hyperparameters: $\alpha$ --- learning rate, $\beta_1, \beta_2 \in [0, 1)$ --- exponential decay rates. It follows algorithm \ref{alg:adam} for the parameter update.
	\begin{algorithm}[H]
		\caption{Adam optimization}\label{alg:adam}
		\item 1. Initialize: $m_0 = 0$ and $v_0 = 0$
		\item 2. Compute gradient: $g_t$
		\item 3. Update biased first moment estimate: $m_t = \beta_1 m_{t-1} + (1 - \beta_1) g_t$
		\item 4. Update biased second raw moment estimate: $v_t = \beta_2 v_{t-1} + (1 - \beta_2) g_t^2$
		\item 5. Compute bias corrected first moment estimate: $\hat{m_t} = \frac{m_t}{1 - \beta_1^t}$
		\item 6. Compute bias corrected second raw moment estimate: $\hat{v_t} = \frac{v_t}{1 - \beta_2^t}$
		\item 7. Apply update: $\theta_{t+1} = \theta_t - \alpha \frac{\hat{m_t}}{\sqrt{\hat{v_t} + \epsilon}}$
		Initial hyperparameters used in this work are $\alpha = 0.001$, $\beta_1 = 0.9$, $\beta_2 = 0.999$ and $\epsilon = 10^{-8}$.
	\end{algorithm}
\end{definition}

\begin{definition}[Overfitting]
	Overfitting is a phenomenon in which a hypothesis that fits training samples well will perform worse over the entire distribution on data rather than another hypothesis that fits the distribution of the training samples less well (\cite{mitchell_1997}). The way to avoid overfitting that happened to the models in Section \ref{section:er} are discussed in Section \ref{section:regularization}.
\end{definition}

\begin{definition}[Feedforward fully connected layer]
	A feedforward fully connected layer is a trainable function with parameters $W \in \mathbb{R}^{N \times M}$ (weights) and $b \in \mathbb{R}^{M}$ (biases) that, in this case, maps a vector $x \in \mathbb{R}^{N}$ to an output $a \in \mathbb{R}^{M}$ via the following transformation:
		\begin{equation}
			a = W^{T}x + b
		\end{equation}
\end{definition}

This is one of the simplest layers in a feedforward neural networks and input and output in it as mentioned above are vectors. However, in this study inputs and outputs are images, that are represented in memory as square matrices $x^{(i)}, y^{(i)} \in \mathbb{R}^{N \times N}$. One could simply flatten the image into a vector and use it as an input to a fully connected feedforward neural network. Nevertheless this would be a suboptimal approach. 

Since essentially one of the main tasks of this research is to create a deep learning model that is able to predict a fluorescence image from a DIC image, the problem statement could be narrowed down to the following: predict an intensity high-resolution image from another intensity high-resolution image based on the features of the object morphology in it. Such problem is very common in the field of image analysis and one of the popular deep learning tools for solving such problems is convolutional neural network (CNN) or more specifically a UNet.

CNNs are able to capture nonlinear relationships over large areas of images, they greatly improve performance for image recognition tasks in comparison to classical machine learning methods (\cite{Ounkomol_2018}). The word "convolutional" suggests that the convolution operation should be used in at least one of the layers there.  

\begin{definition}[Convolutional layer]
	A convolutional layer is a trainable function with parametrized kernel $K \in \mathbb{R}^{F \times F \times C}$ and bias $b \in \mathbb{R}$ that is usually denoted via the operator $(\cdot * \cdot)$. By transforming an input $x \in \mathbb{R}^{W \times H \times C}$ it produces an output $S$
	\begin{equation}
		S = K * x + b
	\end{equation}

	that is called a \textit{feature map} where an element on position $(i, j)$ is defined as follows:
		\begin{equation}
			S_{i, j} = \sum_{w} \sum_{h} x_{m, n}  K_{i - m, j - n}
		\end{equation}
\end{definition}

Convolutional layer like a fully connected layer can be viewed a linear transformation as well. However, there are three main advantages that leverage convolutional layers for image processing in comparison to fully connected layers: sparse interactions, parameter sharing and equivariant representations. An image is a very redundant way of representing the semantic meaning hidden within it. Having a value of one pixel, the probability that the neighboring one will be of the same color is very high. Sparsity of interactions can be described by an example: usually a high-resolution image might have millions of pixels, however it is possible to detect smaller and very important features like contrast changes, edges, and shapes using a kernel consisting of only a few hundred pixels. By applying kernels (or filters) on the image locally, one will infer many of these features across the whole image. Such an approach reduces the memory needed for parameter storing and improves its statistical efficiency (\cite{Goodfellow_2016}). Parameter sharing refers to the fact that instead of learning a separate set of parameters for every location within the image, only one set of parameters will be learned and applied across all image locations. Lastly, equivariance here means that convolution operation is equivarient to the shifts in the image.

\begin{definition}[Stride]
	During the computation of convolution, the kernel starts sliding at the upper left corner of the input tensor, covering all locations while heading to the right and down. The step with which the window slides is called \textit{stride}. 
\end{definition}

\begin{definition}[Padding]
	When convolution is applied several points on the perimeter of the input tensor will be lost and the ouput tensor will have smaller spatial dimension than the input one. One can fix this by adding a few more pixels outside the perimeter, to preserve the dimension of the output to be same as input. The amount of pixels added is called \textit{padding}. 
\end{definition}

Visual examples of what stride and padding represent are shown in Figure \ref{fig:stride}.
\begin{figure}[htb]
	\begin{center}
		\includegraphics[width=0.6\linewidth]{bilder/stride_padding.png}
		\caption[Stride and padding example]%
		{Stride and padding example. Taken from \cite{stride}.}
		\label{fig:stride}
	\end{center}
\end{figure}

\begin{definition}[Max-pooling layer]
	Maximum pooling operation reports the maximum output within a rectangular neighborhood (\cite{Goodfellow_2016}).
\end{definition}

\begin{definition}[Activation function]
	An activation function is an element-wise non-linear function $f(\cdot)$. Some examples are:
	\begin{align}             
		f(x) = \frac{1}{1 + e^{-x}} &&\text{Sigmoid} \\      
		f(x) = max(0, x) &&\text{Rectified linear unit (ReLU)}\\
		f(x) = \begin{cases}
				x, \hspace*{1cm} \textrm{if } x > 0 \\
				\alpha * (e^{x} - 1), \textrm{if }  x \leq 0
		  	\end{cases}\ &&\text{ELU}
		\end{align}
\end{definition}

It is important to use activation functions after each convolutional or linear layer like RELU, ELU, Tahn, Sigmoid or any other non-linearities. Because any combination of linear functions can be represented with another linear function, having consecutive linear layers without non-linear function in the network is equivalent to having just one linear layer. Non-linearities  In CNNs they are also often combined with max-pooling layers and dropouts to escape overfitting. 

\begin{definition}[Batch normalization layer]
	Let's denote $B = \{x^{(i)}, ..., x^{(m)}\}$ to be a mini-batch of data. Then batch normalizing transform applied to this input data would be:
	\begin{equation}
		\begin{split}
		& a^{(i)} = \gamma \frac{x^{(i)} - \mu_B}{\sigma^2_B + \epsilon} + \beta \\
		& \sigma^2_B = \frac{1}{m} \sum_i^m (x^{(i)} - \mu_B)^2 \\
		& \mu_B = \frac{1}{m} \sum_i^m x^{(i)} \\
		\end{split}
	\end{equation}
	where $\gamma$ and $\beta$ are learnable parameters, $\mu_B$ and $\sigma^2_B$ are the mean and standard deviation of the batch (\cite{Ioffe_2015}).
\end{definition}

\begin{definition}[Dropout layer]
	Dropout is a technique that randomly sets some weights (units) to zero (\cite{Srivastava_2014}). It leads to the training of several smaller networks that share the parameters. If a mask vector $\mu$ specifies which units are included in training, then dropout's objective to be minimized becomes: $\mathbb{E}_\mu J(\theta, \mu)$. Visually dropout is presented in the Figure \ref{fig:dropout}.
\end{definition}

%TODO add figure reference!
\begin{figure}[H]
	\begin{center}
		\includegraphics[width=0.5\linewidth]{bilder/dropout.png}
		\caption[Dropout]%
		{Dropout. Taken from \cite{Srivastava_2014}}
		\label{fig:dropout}
	\end{center}
\end{figure}

The models in this project mostly use ELU activations as ELU provides a better signal flow between the layers by not cutting off the negative values completely.

\begin{definition}[UNet]
	UNet is fully convolutional neural network with U-shaped encoder-decoder network architecture (\cite{Ronneberger_2015}). Example of the UNet architecture can be found in Figure \ref{fig:unet}.
\end{definition}

The encoder is a common CNN, consisting of the repeated
block of two $3 \times 3$ convolutions, followed by
an activation function, and a $2 \times 2$ max-pooling operation with stride 2. At each encoder step  the number of feature channels doubles. The decoder is also a CNN, consisting of repeated blocks of transposed convolution, that halves the number of feature channels, followed by a concatenation with a corresponding output from an encoder, and two $3 \times 3$ convolutions, followed by a ReLU. The last decoder layer is a $1 \times 1$ convolution to map the tensor to the number of output image channels needed. Skip-connections is a very important part of UNet as they allow to the flow of high-resolution features from the encoder to the decoder that in turn allows to restore a corresponding high-resolution image.

\begin{definition}[Autoencoder]
	Autoencoder is an unsupervised learning technique in neural networks for the representation learning purposes. Autoencoder consists of an encoder that compresses data into a lower dimensional representation and a decoder that restores the original input from the encoded representation.
\end{definition}

\paragraph{Regularization techniques}
\label{section:regularization-theory}
% TODO add overfitting image?
Regularization is mostly used to prevent a deep learning model to overfitting on the training data and to be able to generalize well. Overfitting has occured in the models used in this research and therefore it is improtant to understand the techniques that can be used to prevent it. There are are several approaches to regularize the model and they will be explained below.

\begin{itemize}
	\item Early-stopping

	Overfitting can be detected via visualizing train and validation losses. Training behaviour at first will be the usual one, meaning that both train and validation losses are gradually decreasing, however at some point the train loss continues to decrease, whereas the validation loss suddenly starts to increase. Since the model has not seen any of the data from the validation set, it means that it loses its ability to generalize on unseen data, while improving its perfomance on the seen data (train set). This does not happen during earlier epochs. Assuming that the model learns a complex decision surface while training, the weights of the model will be quite small and random with the correct weight initialization and therefore the best decision surface during the early epochs would be a smooth one. But during the later ones the difference in values of the weights grows and they become dissimilar which also means that the decision surface becomes more complex and the model is now able to fit not only the training data itself, but also its noise (\cite{mitchell_1997} p.111). And that is why stopping before the model becomes too complex, meaning to stop before the overfitting point, mitigates this problem.

	\item \emph{L1}- \emph{L2}-regularization

	The complexity of the deep model grows with the number of features it uses, sometimes the model may pay attention to the features that are not important to the outcome, or even considers noise to be a feature. To prevent this one should decrease the weights associated with useless features, however one cannot know ahead of time which of them should be ignored, therefore one may limit them all (\cite{Ying_2019}). In order to do that, a penalty term is added to the loss function:

	\begin{equation}
	\tilde{L}(Y, M(X, \theta)) = L(Y, M(X, \theta)) + \lambda R(\theta)
	\end{equation}

	for some $\lambda > 0$. This is called a \emph{soft-constraint} optimization. When $R(\theta)$ is of the form $R(\theta) = ||\theta||^2_2 = \sqrt{\sum\limits_i \theta_i^2}$ this is called \emph{L2}-regularization. When it is of form $R(\theta) = ||\theta||_1 = \sum\limits_i |\theta_i|$ this is called \emph{L1}-regularization. \emph{L2}-regularization used in combination with backpropagation is equivalent to weight decay. Weight decay is defined by \cite{Hanson_1988} as follows:
	\begin{equation}
		\theta_{t+1} = (1 - \lambda)\theta_t - \alpha \frac{\partial L}{\partial \theta_t}
	\end{equation}

	where $\alpha$ is a learning rate. Weight decay successfully has more effect on the weights along which the gradient change is smaller \cite{Goodfellow_2016}. \emph{L1}-regularization induces sparsity of the weights by assigning some of them to zero, this could also be considered as a feature selection approach.

	\item Regularization layers

	Batch normalization and dropout layers are also considered to be a form of regularization.

	\item Network reduction

	Since learning a too complex and noise-fitting decision surface might be a frequent cause of an overfit, another way to mitigate this would to be reduce the space of the possible decision surfaces and therefore make the surface simpler so that it cannot fit into the noise from the data. By changing the number of adaptive parameters in the network, the complexity can be varied (\cite{Bishop_2006} p.332).

	\item Expansion of the training data

	For a successful training a model needs to have a sufficient amount of quality samples. An expanded dataset can improve the quality of the predictions \cite{Ying_2019}, however only when the model has already performed well on the initial dataset. If the model was performing badly initially, adding more data will not solve the problem.
\end{itemize}

        \subsubsection{Dimensionality reduction methods}
            \paragraph{UMAP}
\paragraph{PCA}
\paragraph{PacMAP}
        \subsubsection{Clustering methods}
            After visualizing the embeddings, there is an interest in checking whether they form any kind of clusters. This question is discussed in Section \ref{section:unet-embeddings-dim-reduction} using the DBSCAN algorithm. This part will provide the theory needed to understand how this algorithm works and how it can be set up.
\paragraph{DBSCAN}
\label{section:dbscan}
The DBSCAN is a density-based unsupervised algorithm for discovering clusters. It considers regions with high-density of points to be clusters and points that are located far away from any cluster or form a low density region to be ourliers. This algorithm uses two hyperparameters = \{\textit{eps}, \textit{min\_samples}\} to define the clusters. The first is the distance threshold, which is used to determine whether a point is located in the neighborhood of the other point. The latter one is the minimum number of points that are needed to form one cluster. It splits the points into four categories based on these hyperparameters: \textit{core point} (\textit{min\_samples} points are reachable from it), \textit{directly reachable point} (is within distance \textit{eps} from any core point),  \textit{reachable} (there is a path from a core point to it via directly reachable points) and \textit{outliers}. For more information on how clusters are formed refer to \cite{dbscan}. DBSCAN does not require the provision of the number of clusters in advance. Although this is a nice quality of this algorithm it is not that important for current research as the number of clusters that is needed here is known in advance. For example, this algorithm is used in Section \ref{section:unet-embeddings-study} in order to check whether different phenotypes form different clusters or, for example, whether corrupted images would fall into a separate clusters. In all cases the number of ground truth clusters is known in advance.  
    \subsection{Imaging}
        \subsubsection{Digital imaging}
            Digitally an image is represented as an array of size $(H, W, C)$ where $H$ is the height, $W$ is the width and $C$ is the number of channels of the image. In this work, $C = 1$ and $W = H$. A digital image $A$ can be represented represented with the matrix:

\begin{equation}
    A = \left[
            \begin{array}{ccc}
                a_{0,0} & \cdots & a_{0,W-1} \\
                \vdots & \ddots & \vdots \\
                a_{H, 0} & \cdots & a_{H-1, W-1}
            \end{array}
        \right]
\end{equation}
where $a_{i, j} \in \mathbb{R}$. Both DIC and fluorescence images were provided in tag image file format (TIFF). For the processing convenience purpose all images were normalized to be in the range of $[0, 1]$:

\begin{equation}
    a^{\text{norm}}_{i,j} = \frac{a_{i, j} - \min(A)}{\max(A) - \min(A)}
\end{equation}
for $ \forall i \in \{0, ..., W - 1\}$ and $ \forall j \in \{0, ..., H - 1\}$
            How image is stored in memory, which conventions there are (RGB, BGR (conventions are used in corruptions augmentations)).
        \subsubsection{Microscopy imaging}
            \paragraph{Image acquisition peculiarities} 
    Cells used in this research are growing in 96-well plates. A plate or a microplate in biology is a flat plat with multiple tubes ("wells"). The microscope used in the experiments takes photos of the well plate in random locations. The reason for that hides in the settings for focuing in microscope. To get a reasonably good photo without blur it has to focus on a specific location of the plate, the choice of the location however happens automatically, therefore the location of the focus is random (see Figure \ref{fig:random-dic}). 
    
    Unfortunately it might be problematic in the following sense: photos takes in such manner do not gurantee that the focus will land in distinct spots all the time. Meaning that some cells present in one of the photos might appear in the other ones. Since the photos have a high-resolution they are first splitted into crops of size $256 \times 256$ each. It might happen that same cells might appear in several crops. That is why after split of the image data between train, test and validation sets it might be that the same set of cells will once land in the train set and another time in the validation set, which will lead to a not completely fair and representative validaiton loss during training.
    
    In order to overcome this problem a much more expensive equipment is needed. Since in our case it doesn't bring too huge problems expect for the fact that validation metrics might be lower than they should have been, there was no need to purchase a more expensive equipment.   
    
    \begin{figure}[htb]
        \begin{center}
            \includegraphics[width=0.3\linewidth]{bilder/dic-random.png}
            \caption{Way in which photos of the well-plate were taken}\label{fig:random-dic}
        \end{center}
    \end{figure}    
\paragraph{Crops combination technique}
    \input{content/domain knowledge/crops combination.tex}
\pagebreak
\section{Implementation and experiments}
In this chapter the results of all experiments performed for predicting fluorescence signal from four cell organelles: nuclei, endoplasmic reticulum, Golgi apparatus, and full cell fluorescence are provided and discussed. This part first starts out with a description of the models and data used in the experiments, followed by four subsections dedicated to each of the organelles. Each subsection describes its own different approaches in pre- or postprocessing needed, difficulties that occurred during preparation or training steps as well as the results obtained for each organelle separately.

    \subsection{Model training}
    \subsubsection{Neural network architecture}
        \input{content/nn architecture.tex}
    \subsubsection{Available data}
        \input{content/data.tex}
    \subsubsection{Training costs estimation}
        \input{content/costs and times.tex}
    \subsubsection{Augmentations}
        \label{section:augmentations}
        \input{content/model training/augmentations.tex}
    \subsubsection{Model setup}
        \paragraph{Weight Initialization}
        In order to achieve best predictions results it is very important to pre-setup a model correctly. Since the architecture used here is very similar to the one used in LaChance paper, the setup configuration is similar as well.  
        \input{content/model training/wi.tex}
        \paragraph{Regularization}
        \input{content/model training/regularization.tex}
        \paragraph{Optimizers}
        \input{content/model training/optimizers.tex}

    \pagebreak
    \subsection{Nuclei}
    A nucleus (plural nuclei), as related to genomics, is the membrane-enclosed organelle within a cell that contains the chromosomes. The nucleus is one of the easiest organelle to detect within the cell as it is usually located in the middle and occupies quite a big area of the cell (see Figure \ref{fig:cell}). Nucleus contains all of the cell's chromosomes, which in their turn encode the genetic material, therefore nucleus is a very important organelle (\cite{genomegov}). In order to stain it, DAPI was added to the cells. This is a fluorescent stain that binds strongly with some regions in DNA. Analysis of cell's nucleus can provide many valuable insights, for example the radius of living cells is on average bigger then in dead ones (\cite{Christiansen_2018}). With fluorescence labeling one can derive some useful features used for determining whether the well plate should or should not be selected during the selection step in CLD process.
    %cite Christiansen 2018]  proves only success on nuclei for DIC
    \begin{figure}[htb]
        \begin{center}
            \includegraphics[width=0.3\linewidth]{bilder/cell structure.png}
            \caption{Cell structure}\label{fig:cell}
        \end{center}
    \end{figure}

    \subsubsection{Preprocessing}\label{section:nuclei-preprocessing}
        \input{content/nuclei/preprocessing.tex}
    \subsubsection{Training and predictions}
        \paragraph{Convergence}
              \input{content/nuclei/convergence.tex}
        \paragraph{Predictions quality}
              \input{content/nuclei/quality.tex}
    \subsubsection{Postprocessing for nuclei segmentation}
        \input{content/nuclei/postprocessing.tex}
        \paragraph{Thresholding algorithms}
        \input{content/nuclei/thresholding.tex}
    \subsubsection{Influence of scaling on predictions quality}
        \input{content/nuclei/scaling influence.tex}
    \subsubsection{Conclusions}
        \input{content/nuclei/summary.tex}
    \pagebreak
    \subsection{Endoplasmic Reticulum}
    \label{section:er}
    Endoplasmic reticulum (ER) is a network of membranes inside a cell, through which proteins and other molecules move. Ribosomes are small and round organelles with the main function to produce the protein needed for the cell. They are located in a continuous membrane system that forms series of flattened sacs, this membrane is called ER (see Figure \ref{fig:cell}) (\cite{er}). ER itself is moslty located around the nuclei. In order to stain ER, CHO cells were treated with Donkey Anti-Rabbit IgG antibody. This is a fluorescent stain that binds strongly with ER. The analysis of this cell organelle is also important for the CLD as ER is directly related to the process of protein production within the cell: it is responsible for the synthesis, folding, modification, and transport of proteins (\cite{er_2}). The proximity of the ER to the nucleus essentially allows to control the protein production. For example, when the protein is incorrectly folded it will be accumulated in the ER lumen and it will serve as a signal to activate misfolded protein response from the cell. In contrast to other imaging datasets ER dataset does not require special preprocessing steps as the images are of a good quality and without any visible problems (see the ground truth image in Figure \ref{fig:er-combined}).
    
    \subsubsection{Training and predictions}
        \input{content/er/convergence.tex}
    \subsubsection{Combination of nuclei and actin predictions}
        \input{content/er/nuclei combined.tex}
    \subsubsection{Postprocessing for ER segmentation}
        \input{content/er/postprocessing.tex}
    \subsubsection{Biological metrics}
        \input{content/er/downstream metrics.tex}
    \subsubsection{Conclusions}
        \input{content/er/summary.tex}
    
    \pagebreak
    \subsection{Golgi apparatus}
    The Golgi apparatus (or Golgi, or Golgi complex) is another organelle inside the cell that packages proteins into membrane-bound vesicles which will be exported from the cell. Golgi is also located near the nucleus and ER as well, it is even called a perinuclear body. This location is explained by the biological processes: after a protein comes out of the ER, it goes into the Golgi for further processing (\cite{golgi}). The staining of Golgi apparatus has turned out to be the most difficult of all. Two antibodies were tried out in the beginning, however both of them resulted in a low signal-to-noise ratio. Many images were underexposed, and the density of the cells after their fixation was pretty low. The hypothesis why this was the case was that the choice of target protein in Golgi to which antibody can bind to was not the best. Golgi appartus represents an interest for this study, because it is a very difficult fluorescence target. Even having a fluorescence imaging with high signal-to-noise ratio, the lowest scores among all cell organelles in state-of-the-art paper \cite{Cheng_2021} were achieved on Golgi.
    \subsubsection{Preprocessing}
        \input{content/golgi/golgi preprocessing.tex}
        \input{content/background removal.tex}
    \subsubsection{Training and predictions}
        \input{content/golgi/convergence.tex}

    \subsubsection{Alternative ways to improve predictions}
        \paragraph{Asymmetrical losses}
            \input{content/golgi/asymmetrical training.tex}
    \subsubsection{Conclusions}
        \input{content/golgi/conclusions.tex}
    \pagebreak
    \subsection{GFP}
    \paragraph{Preprocessing}
        \input{content/gfp/preprocessing.tex}
    \paragraph{Predictions}
        \input{content/gfp/predictions.tex}
    \paragraph{Downstream metrics}
        TODO move to separate chapter?
        \input{content/gfp/postprocessing.tex}
    \paragraph{Combination of GFP, nuclei and ER}
        \input{content/gfp/combination.tex}

\pagebreak
\section{Stability study}
    
\subsection{Corruptions}  
    For practical reasons it is important to not only evaluate the models on high-quality data exclusively, but also to know how the predictions will degrade when the input's data quality decreases. Having a model for fluorescence \textit{in silico} labeling that can additionally alarm end users when the predictions should not be relied upon is very useful in practice. Although the DIC microscopy is a relatively easy technique, there are still setting up procedures taking place that can be prone to errors. Additionally, as the models are not easily generalizable across phenotypes as well as between fixed and not fixed cells, an alarming system that is able to catch these situations would be useful to save time and cost of lab work. In order to measure the stability or robustness of the models towards data degeneration they were evaluated on the corrupted or "bad" input DIC images. There are two sources of "bad" images that can be used for such estimations. The first are actually corrupted images made in the laboratory. Such corruptions may come from different sources: for example, an oil bubble landed on the microscope lenses, low density of the cell on the image, over- or underexposure during image acquisition. Another source of image corruption would be images with artificial or pseudocorruptions created manually via image processing. They allow more systematic investigation of the impact of a corruption effect. Artificial corruptions allow to vary the severity of the corruption keeping the original fluorescence data intact.
    
    %This chapter first provides a description of artificial corruptions used to evaluate previously trained models on. Afterwards the real examples of corruptions acquired from the lab are 
    \subsubsection{Artificial corruptions}
        \input{content/stability/artificial corruptions.tex}
    \subsubsection{Real corruptions}
        \label{section:real-corruptions}
        \input{content/stability/real corruptions.tex}
    \subsubsection{Improving predictions with additional corruption augmentations}
        \label{section:augments-againts-corruptions}
        \input{content/stability/augmentations.tex}
    \subsubsection{Generalizability across phenotypes}
        TODO train the model on one phenotype and predict on the other, compare predictions (visually?)
        postprocessing with metrics then?
    \pagebreak
    \subsection{UNET embeddings study}
    Similarly to studying autoencoder embeddings that represent a high-dimentional input in lower dimentional space, one can study UNet embeddings. However, it is important to keep in mind that the dimensionality of embeddings in UNet case is not lower than dimensionality of the input and is even often higher (see the UNet architecture in Figure \ref{fig:unet}). The goal of a UNet in contrast to an autoencoder is not to compress the input, but to extract useful features that are helpful for high-resolution segmentation. UNet embeddings do not contain rich image semantics in them as embeddings of an autoencoder do. UNet compresses the spatial dimention of the input, but at the sme time it gradually increases the number of filters that capture of information need for segmentation. As it has been proven in section \ref{section:nuclei-predictions} having more filters only helps to get better predictions, therefore there is no need for a UNet to have low-dimentional embeddings. Nevertheless, it is still interesting to see if the embeddings do contain any information about the input that one could use. There were two hypothesis put in question: the first one is whether embeddings of a trained UNet form any kind of clusters based on cells phenotype. And a second one is whether embeddings of corrupted images can be clustered together further away from not corrupted ones. If the latter hypothesis would hold, one could alarm the end-user about the outliers in the dataset based on their distance from both of the clusters. 
    \subsubsection{Application of various dimentionality reduction methods}
        \label{section:unet-embeddings-dim-reduction}
        \input{content/unet-embeddings/unet-embeddings.tex}
    \subsubsection{Autoencoder embeddings as an alternative}
        \input{content/autoencoder.tex}
    \subsubsection{Clustering of PacMAP embeddings}
        \paragraph{Clustering on UNet embeddings}
        \label{section:clustering-on-unet-embeddings}
        \input{content/unet-embeddings/unet-embeddings-clustering.tex}
    \subsubsection{Embeddings for direct stability prediction}
        \input{content/unet-embeddings/stability prediction.tex}
    \pagebreak
    \subsection{Drift detection}
    Assume that during training labeled data comes from a distribution $p$, meaning $\{(x^{(1)}, y^{(1)}), ..., (x^{(n)}, y^{(n)})\} \sim p$ and during deployment unlabeled data comes from a distribution $q$, meaning $\{x^{(1)}\prime, ..., x^{(1)}\prime\} \sim q$. The goal of the drift detection is to determine if $q(x\prime)$ is the same data distribution as $p(x)$. Or, putting it more formally, determine which hypothesis holds: null-hypothesis $H_0$ and an alternative hypothesis $H_A$, where $H_0:p(x) = q(x)$ and $H_A:p(x) \neq q(x)$.

    Having samples from both distributions or representation of these samples in lower dimension, one can then choose a statistical hypothesis test to compare these distributions (\cite{Muandet_2017}).
    \subsubsection{Drift detection vs. outliers detection}
        \input{content/drift detection/drift_vs_outliters.tex}
    \subsubsection{Kernel methods and two-sample testing}
        \input{content/drift detection/kernel methods.tex}
    \subsubsection{Maximum mean discrepancy for drift detection}
        \input{content/drift detection/mmd.tex}
    \subsubsection{Drift detection experiments}
        \input{content/drift detection/dd.tex}
    \subsubsection{Online drift detection experiments}
            \input{content/drift detection/online.tex}
            \paragraph{Impact of cell fixation}
                In section \ref{section:gfp} the difference between fixed and not fixed cells was mentioned. Visual analysis of model's predictions for not fixed cells after training it on fixed ones has shown that the model was not able to generalize well on them. This is the reason why it would be important to alarm the end user to not rely on predictions when such a situation occurs. In this case an online drift detector trained using not corrupted data used for ER training first and tested on not fixed ER cells. The results of this test are shown in Figure \ref{fig:online-drift-not-fixed}.
                \begin{figure}[htb]
                    \begin{center}
                        \includegraphics[width=0.5\linewidth]{bilder/drift-detection/online-fixed-vs-not-fixed.png}
                        \caption{Online drift detection of not fixated cells}\label{fig:online-drift-not-fixed}
                    \end{center}
                \end{figure}
                The ERTs for corrupted data (left) are lower from ERT for true input. The ROC-AUC score for the separability is $0.91$ and the best threshold is $6$. However, not corrupted data (fixed cells) mostly have an ERT of $7$, whereas corrupted data (not fixed cells) have an ERT of $4$. Both classes have ERTs that are very close to the threshold, but are able to separate the classes well enough.

                Application of a usual drift detection algorithm with the use of ER model the false positive rate on not corrupted (fixed) cells was $0.075$. Whereas all fixed cells were recognized as drift.  
\pagebreak
\section{Future research}
"
One limitation of our current work is that it is based on fixed cells that does not allow longitudinal imaging. This can be overcome by using fluorescent reporter cell lines or live cell dyes to provide the fluorescence ground truth (10) and enable dynamic observation. Another limitation of the DL framework we used here is that it cannot be generalized to different types of cells. Techniques based on transfer learning (https://doi.org/10.1038/s41551-019-0362-y, https://downloads.spj.sciencemag.org/bmef/2020/9647163.pdf) and domain adaptation (38) will be investigated in our future work to overcome this limitation."
TODO rephrase https://www.science.org/doi/10.1126/sciadv.abe0431
\pagebreak
\section{Summary}
In this master thesis three deep learning models were developed that are able to predict fluorescence signal from DIC image data for the following targets: nuclei, endoplasmic reticulum and full cell surface. Training experiments for Golgi apparatus target have shown that there is a lack of training data with high enough signal-to-noise ratio and further research is required. Three models mentioned above can successfully replace manual fluorescence staining procedure.

Before training the models, several preparation stages have been developed. First, the model's architecture was improved for faster learning with the use of batch normalization layers. Image augmentations such as rotations and scaling were added and their logic was improved through the use of the high-resolution original image. The model was tuned by choosing correct regularization and optimization algorithms (batch normalization and adadelta optimizer in this case). For each of the four targets corresponding image preprocessing  procedures were developed in order to improve data quality and address model's limitations before training (such as background-foreground class imbalance for instance). Advanced background removal and enhancement approaches were developed for the Golgi preprocessing target.

In each training procedure, the model has successfully converged. Intensity fluorescence predictions as well as binary segmentation (for the full cell surface target) were visualized. It was shown that the use of a bigger model and more data significantly improves the predictions (especially total and mean intensities show an immediate improvement in correlation coefficients). For a better models evaluation, more practical downstream metrics were proposed. These are organelle quantity, total intensity, mean intensity and area of the organelle of interest. Every model was evaluated based on these metrics, it was shown that not only a significant correlation with ground truth values in terms of Pearson and Spearman rank correlation coefficients is present, but the forms of distributions visualized with violin plots are very similar. All models have a similar downside in overpredicting total and mean intensities, however strong correlation betwen the values suggests that there is an absolute value shift in predictions that can be relatively easily fixed.

Evaluation of the models on these downstream metrics also required  development of the corresponding segmentation pipelines for every organelle in question. They were successfully built with the OpenCV and \textit{skimage} libraries. 

UNet embeddings were studied for the possible source of additional data insights like phenotype differences or corruptions and unsupervised clustering algorithms were applied. The study has shown that image embeddings are not clustered based on cell phenotype, however corrupted images indeed form a separate cluster in the embeddings space. Nevertheless, this cluster is not well-separable from the rest of the data and further research is required. Autoencoder embedding that was trained on the same data, was checked for the same clustering targets. The results have shown strong clustering based on the brightness of the crops, which is not significant for this research.

Finally, the stability of the models was studied. Two types of image corruptions were introduced: artificial corruptions via image processing and corrupted data from the wrong microscopy settings. It was shown that models are very stable against changes in contrast and brightness, which can be caused by an over- or underexposure. Although the models were very prone to errors with defocus blur corruption, it was shown that including artificial corruptions in augmentations immediately improves the results. A generalizability study of the model across cell phenotypes and across not fixed cells was carried out. Prediction on the image with corruptions created in the laboratory have shown that models are stable against errors in cell fixation process, however they are quite sensitive towards errors in the focus of the microscope. In order to detect corruprions, two drift detection algorithms were developed. The first one is based on the maximum mean discrepancy method, and the second one is an online version of the first one. They were tested on both corruption types and have shown strong ability for detecting drift in data with high ROC-AUC scores.
  
\subsection{Limitations}
The main limitation of this research is the need to fix the cells before taking a DIC image of them. Cell fixation is a preceding step before cell staining --- therefore all cells in datsets of DIC images were fixed. Since living and fixed cells look very different and models trained on fixed cells do not generalize well to not fixed ones, predictions can be done on DIC images only. Luckily, fixing cells is not a cumbersome lab procedure and is far easier than staining the cells, which is avoided with the help of \textit{in silico} fluorescence labeling. It is recommended to look into possibilities of transfer learning from fixed to living cells.  

Also, models developed here cannot generalize well on other cell types and are able to give good predictions for the cells used in this laboratory only. However, in future the developed product would aim to address specific need of each laboratory separately. Therefore the models will be trained with the goal to address the issues of one specific cell line. 


\subsection{Future research}
This research sets the ground for a variety of futher research ideas:
\begin{itemize}
    \item As Golgi apparatus model did not produce good enough predictions due to several reasons, such as low signal-to-noise ratio in input data, as well as an extremely strong class imbalance between the foreground and background in images, it is strongly recommended to continue research in this direction. For example, by choosing another target protein in staining procedure, apppying stronger noise reduction approaches like \cite{noise2void}, and incorporating image gradients in the loss function.
    \item UNet embeddings do indeed show a potential for detecting corruptions in crops, however it is not strong enough. It is highly recommended to incorporate embeddings of all crops from the same image into one point in the embedding space, for example, by averaging the embeddings. This might show a more significant clustering of corrupted data.
    \item Since autoencoder embeddings have shown a clear clustering based on the crop brightness, it would be beneficial to normalize the brightness across all crops first. This would allow an autoencoder to pay attention to other less distinctive features in the image.
    \item Due to the time constraints it was not possible to train very big models, however the improvement of predictions quality that happens after a model enlargement is very significant. Therefore it is recommended to train models with a bigger model size and more data.
    \item The lack of data in corruptions from the microscopy settings did not allow to test online drift detection algorithm. As this version is very promising to be used in practice it is recommended to use more data for its testing.
\end{itemize}
 


%%%%%%%%%%%%%%%%%%%%%%%%%%%%%%%%%%%%%%%%%%%%%%%%%%%%%%%%%%%%%%%%%%%%%%%%
%%%% ENDE TEXTTEIL %%%%%%%%%%%%%%%%%%%%%%%%%%%%%%%%%%%%%%%%%%%%%%%%%%%%%
%%%%%%%%%%%%%%%%%%%%%%%%%%%%%%%%%%%%%%%%%%%%%%%%%%%%%%%%%%%%%%%%%%%%%%%%

\clearpage

% Entfernen Sie das Kommentar aus der nachfolgenden Zeile, falls Sie einen Anhang in der Arbeit verwenden wollen. Beachten Sie, dass Sie sich im Verlauf der Arbeit mit \ref{...} (z.B. \ref{anhang:zusatz1}) auf den Anhang beziehen.
%\newpage
\appendix
\section{Anhang}

\subsection*{Zusatzteil 1} \label{anhang:zusatz1}

Dies ist ein Anhang.

\clearpage

\ifthenelse{\boolean{\biber}}{ %with biber do
	\DeclareNameAlias{sortname}{first-last}
	\printbibliography[heading=bibintoc, title=\references]
}{ %without biber do
	\bibliography{references}
	\bibliographystyle{alphadin}
}
%\vspace*{\fill}

\clearpage

\listoffigures

\listoftables

%\pagebreak

%\printindex
\end{document}
