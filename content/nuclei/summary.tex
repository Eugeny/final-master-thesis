Fluorescence staining of nucleus in CHO cells can be successfully replaced with \textit{in silico} labeling using deep neural network. The size of the dataset is big enough to achieve accurate predictions, however a further research could be dedicated to improving fluoreschence image preprocessing in order to remove the blur around the nucleus, as well as clear the data from the under- or overexposed images. 

The model successfully converges, and the use of augmentations improve the predictions accuracy. It is recommended to use Person correlation coefficient or downstream metrics to evaluate the model performance as MSE is not representative of a model's quality in this case. A further reseach can be also conducted to improve the architecture of the model, as the use of more parameters imrpoves the predictions significantly. 

In order to perform segmentation of the nucleus it is recommended to use the local thresholding approach when the inference time is not crucial and global minimum thresholding algorithm when the inference time is critical. Training of the cells of a larger size slightly imrpoves the intensity accuracy. However slight changes in the cell size do not influence model's performance significantly.