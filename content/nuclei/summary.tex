Fluorescence staining of nucleus in CHO cells used for recombinant protein production at Merck KGaA can be successfully replaced with \textit{in silico} labeling using deep a neural network. The provided dataset is big enough to achieve accurate predictions, nevertheless further research could be dedicated to improving fluorescence image preprocessing in order to remove the blur around the nucleus, as well as to cleaning the data from the under- or overexposed images.

The model successfully converges, and the use of augmentations improves its performance. It is recommended to use Pearson correlation coefficient or biological metrics to evaluate the model's predictions as MSE is not representative of a model's quality in this case. The performance of a model is evaluated in terms of practical biological metrics. Further research can also be conducted to advance the architecture of the model, as the use of more parameters improved the predictions significantly both visually and in metrics.

In order to perform segmentation of the nucleus it is recommended to use a local thresholding approach when the inference time is not crucial and global minimum thresholding algorithm when the inference time is critical. Training the model on cells of a larger size slightly improves the intensity accuracy. Yet a slight change in cell size does not influence the model's performance significantly.
