Following the development phase, when the model training is finished, the model will be moved into deployment or production, where it is supposed to maintain an expected quality of predictions. However, input data is not always a stable source of input. One should constantly maintain quality of predictions and do regular check-ups for outliers as well as to alert the end user about a drift in input data. Drift detection happens on raw data in absence of the ground truth labels and serves as a signal that the input data differs a lot from the data used for training, meaning that predictions became unreliable.

There is a significant difference between distinguishing drift of the whole source of data in comparison to detecting single outliers. In drift detection, one looks at the whole new input data as a distribution and checks if there is a significant shift in comparison to the data used during training. To compare original training data distribution with the new one from inputs different statistical tests like Kolmogorov-Smirnov, Chi-squared and others can used.

There are two possible reactions after the drift is detected: alert the user that predictions became unreliable, and and therefore the expansion of the dataset should be considered by adding more labeled data from a newly drifted distribution in training, or applying some different logic on the model outputs. When an outlier is detected, a model might request human assistance for some particular input, because this input is too unfamiliar to the model and possibly it will not return good predictions on this one (\cite{samuylova_2021}). 

In summary, the drift detection is needed only when the meaningful shifts of the input data distribution from the training distribution need to be detected. Which is exactly the case in this project. During production there can be various shifts in image acquisition processes due to the different microscopy settings and it is beneficial to be able to detect them. The models are trained assuming the correct setup of microscopy image acquisition, however changes in exposure, illumination, cell fixation procedure might alter DIC imaging. In this case the user has to be informed about it and choose afterwards whether more data should be added to the training set or whether the mode's predictions should not be used.

It is also worth noticing that since the images are split into crops and fed into the drift detection model by parts, several crops from image can already be representative of the new image distribution. Therefore if the drift detection model is fast enough to detect drift based on the few crops only, the detection will happen after the crops collected from one image only, such a model can be used as an outlier detector.
