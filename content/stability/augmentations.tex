Going from observations of the models' stability towards different image brightness which is present in the datasets a new hypothesis was drawn. Introducing corruptions that we test on into the training should improve the predictions on corrupted data. Unfortunately, it is not possible to use real lab corruptions here as the data was provided only for testing on these difficult cases and was not stained. Without staining one cannot give a quantitative measure of the quality of the predictions. However, artificial corruptions can be applied here easily. Random changes in contrast, brightness and defocus blur of severity levels $-4$ and $4$ were added to the training augmentations. After the improved model was trained the predictions on the corrupted dataset became much better indeed (see Figure \ref{fig:augments-help}).

\begin{figure}[htb]
	\begin{center}
		\includegraphics[width=0.4\linewidth]{bilder/stability/augments-help.png}
		\caption{Using corruptions as augmentations to improve predictions predictions}\label{fig:augments-help}
	\end{center}
\end{figure}

\subsubsection{Influence of corruptions on metrics for downstream tasks}
Additionally, since for artificial corruptions the ground truth data from staining is present, the difference in downstream metrics for models with and without augmentations was measured (see Table \ref{table:nuclei-corruptions-downstream-metrics-coefficients}), more specifically Spearman rank correlarion coefficients are compared. The calculation of downstream metrics remained the same except the thresholding algorithm was switched to a global one due to the time limit. This results in the wrong segmentation of some of the ground ground truth images. Therefore Spearman rank correlation coefficient is more repesentative here since it is more stable towards the outliers. The rest of the postprocessing procedure has remained the same apart from the application of artificial corruptions on the input data.

\begin{table}[htb]
    \centering
    \caption{Correlation coefficients for downstream tasks on nuclei}
        \begin{adjustbox}{width=0.7\linewidth}
            \begin{tabular}{|c|c|c|c|c|}\hline
                Contrast level&Number of nuclei&Total intensity&Mean intensity&Area\\\hline\hline
                +1&0.934&0.825&0.826&0.898\\\hline
                +2&0.932&0.820&0.819&0.899\\\hline
                +3&0.934&0.799&0.822&0.890\\\hline
                +4&0.804&0.439&0.671&0.540\\\hline
                +5&0.394&0.351&0.383&0.313\\\hline \hline
				Defocus blur level&Number of nuclei&Total intensity&Mean intensity&Area\\\hline\hline
                +1&0.934&0.832&0.827&0.905\\\hline
                +2&0.929&0.800&0.820&0.890\\\hline
                +3&0.934&0.756&0.8210&0.871\\\hline
                +4&0.838&0.361&0.666&0.501\\\hline
                +5&-0.072&-0.233&-0.231&0.07\\\hline
            \end{tabular}
        \label{table:nuclei-corruptions-downstream-metrics-coefficients}
        \end{adjustbox}
\end{table}

One can observe from the table above that the metrics degrade quite fast starting from the severity level $3$. The most affected metrics are the total and mean intensity. The number of organelles seems to be the most stable one until the very last severity level when the predictions turn almost completely black. 