Golgi apparatus the most difficult target in this research and the models trained on Golgi data require futher research. The difficulty in predictions of Golgi is explained due to the low signal-to-noise ratio of the acquired ground truth fluorescence imaging, overall low density of the cell in images and the low quality of the antibodies chosed for cell staining. 

Golgi imaging requires a strong preprocessing such as enhancement and background removal with rolling ball algorithm due to the presence of non-specific fluorescence lightning. However even these approaches do not significantly improve the quality of training data and the predictions are still not accurate. It is recommened to use a \textit{noise2void} denoising algorithm called (\cite{noise2void}) as well as explore other losses that would include the granularity of the Golgi stains, for example a loss that uses Sobel filters.