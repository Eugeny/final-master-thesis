The Golgi apparatus is the most difficult target in this research and the models trained on Golgi data require further research. The difficulty of Golgi predictions can be attributed to the low signal-to-noise ratio of the acquired ground truth fluorescence imaging, overall low density of the cell in images and the inherent challenges of antibody staining of the Golgi apparatus. 

Golgi imaging requires a strong preprocessing such as enhancement and background removal with rolling ball algorithm due to the presence of non-specific fluorescence lighting. However, even these approaches do not significantly improve the quality of training data and the predictions are still not accurate. \cite{Cheng_2021} put a special attention on the significant importance of a good signal-to-noise ratio for quality Golgi fluorescence predictions. The aforementioned paper uses an advanced denoising algorithm called \textit{noise2void} (\cite{noise2void}). This is a strong method that does not require pairs target clean signal and noisy images available, but simply assumes that that there are only noisy images available. They use the idea that if one could acquire multiple images with the same signal but with different realizations of noise, then after averaging across these images the resulting image would approach the true signal. This might be a useful approach for denoising Golgi imaging instead of the use of enhancement and rolling ball approaches. It is highly advisable to try this method in future research. However the initial exploration of this method has shown that it is very expensive in terms of the computational costs. 

One of the insights from the predictions is that there are not enough details inside the predicted fluorescence. The texture and the gradient seem to be missing and predictions are too smooth. Although this is not very crucial for other organelles, Golgi apparatus stain on the other hand has a very granular structure and consists of small dots. In order to introduce the granularity we suggest to try the introduction of a gradient of an image into the loss function. For example, by calculating the image gradient with Sobel filters for ground truth and prediction and incorporating the difference into the loss function (see \cite{Yao_2016}).