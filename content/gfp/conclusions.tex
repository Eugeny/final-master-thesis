GFP \textit{in silico} fluorescence labeling can successfully be used for counting the number of cells and estimating their size. The created models can both predict the intensity images as well as binarized masks. Intensity predictions might be more useful for the number of cells estimation as the cells better visually separate there. The models were evaluated on two downstream metrics: cell size and cell count. The correlation coefficient suggests a strong correlation between the predictions and ground truth. The model has limitations in its ability to differentiate between dead and alive cells. This issue should be addressed in further reasearch after acquiring data for labeling dead cells. Despite the difficulty to clearly determine cell boundaries during image preprocessing for fluorescence image binarization (its overprediction in some cases) the model can successfully generalize and predict a correct boundary for all cells. The images of not fixed cells were provided for the first time during GFP experiments and it was clear that previously trained models do not generalize on them well.  Because of this the limitation of research in terms of the cell fixation need was determined. However, it is recommended to look into possible transfer learning approaches get rid of the fixation step completely.
  