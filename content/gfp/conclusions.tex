GFP \textit{in silico} fluorescence labeling can successfully be used for counting number of cells and estimating their size. The created models can both predict the intensity images as well as binarized masks. Intensity predictions might be more useful for number of cell estimation as the cells separate there visually better. The models was evaluated on two donstream metrics: cell size and cell count. Correlation coefficient suggest a strong correlation between the predictions and ground truth. The model has limitation in its diability to differentiate between dead and alive cells. This issue should be addressed in further reasearch after acquiring data for labeling dead cells. Despite the difficulty of clear determination of cell boundary during image preprocessing for fluorescence image binarization (its overprediction in some cases) the model can successfully generalize and predict a correct boundary for all cells. The images of not fixed cells were provided for the first time during GFP experiments and it was clear that previously trained model do not generalize on them well, therefore the limitation of research in terms of the cell fixation need was determined. However, its recommened to look into possible transfer learning approaches to get rid of fixation step completely.
  