After visualizing the embeddings, there is an interest in checking whether they form any kind of clusters. This question is discussed in Section [TODO cite the section] using DBSCAN algorithm. This part will provide the theory needed to understand how this algorithm works and how it can be set up.
\paragraph{DBSCAN}
A density-based algorithm for discovering clusters (DBSCAN) does not require the provision of the number of clusters in advance. Although this is a nice quality of this algorithm it is not that important for current research. The goal of Section [cite Section] is to check whether different phenotypes form different clusters or, for example, whether corrupted images would fall into a separate clusters. In all cases the number of ground truth clusters is known in advance. However, the fact that corrupted images may form a denser region in their embeddings in comparison to non-corrupted ones [cite Section] may be an indicator that exactly DBSCAN would give a good clustering result. 

This algorithm uses two hyperparameters = \{eps, min\_samples\} to define the clusters. The first is the distance threshold, which is used to determine whether a point is located in the neighborhood of the other point. The latter one is the minimum number of points that are needed to form one cluster. DBSCAN clusters points not only into several clusters but also determines the points that could not be assigned to any cluster, which is also very useful in this research.