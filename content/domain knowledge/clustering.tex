After visualizing the embeddings, there is an interest in checking whether they form any kind of clusters. This question is discussed in Section \ref{section:unet-embeddings-dim-reduction} using the DBSCAN algorithm. This part will provide the theory needed to understand how this algorithm works and how it can be set up.
\paragraph{DBSCAN}
\label{section:dbscan}
The DBSCAN is a density-based unsupervised algorithm for discovering clusters. It considers regions with high-density of points to be clusters and points that are located far away from any cluster or form a low density region to be ourliers. This algorithm uses two hyperparameters = \{\textit{eps}, \textit{min\_samples}\} to define the clusters. The first is the distance threshold, which is used to determine whether a point is located in the neighborhood of the other point. The latter one is the minimum number of points that are needed to form one cluster. It splits the points into four categories based on these hyperparameters: \textit{core point} (\textit{min\_samples} points are reachable from it), \textit{directly reachable point} (is within distance \textit{eps} from any core point),  \textit{reachable} (there is a path from a core point to it via directly reachable points) and \textit{outliers}. For more information on how clusters are formed refer to \cite{dbscan}. DBSCAN does not require the provision of the number of clusters in advance. Although this is a nice quality of this algorithm it is not that important for current research as the number of clusters that is needed here is known in advance. For example, this algorithm is used in Section \ref{section:unet-embeddings-study} in order to check whether different phenotypes form different clusters or, for example, whether corrupted images would fall into a separate clusters. In all cases the number of ground truth clusters is known in advance.  