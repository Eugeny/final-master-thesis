After visualizing the embeddings, there is an interest in checking wether they form any kind of clusters, this question is discussed in the Section [TODO cite the section] using DBSCAN algorithm. Here will be provided the theory needed to understand how this algorithm works and how it can be set up.
\paragraph{DBSCAN}
A density-based algorithm for discovering clusters (DBSCAN) does not require to provide the number of clusters in advance. Altough this is a nice quality of this algorithm it is not that important for current research. The goal of Section [cite Section] is to check wether different phenotypes form different clusters or for example wether corrupted images would fall into a separate clusters. In all cases the number of ground truth clusters is known in advance. However the fact that corrupted images form a more dense region in rather than non corrupted ones [cite Section] is an indicator that exactly DBSCAN would give a good clustering result. 

This algorithm uses two hyperparameters = \{eps, min\_samples\} to define the clusters. The first is the distance threshold, that is used to find determine wether a point is located in the neighborhood of the other point. The latter one is the minimum number of points that are needed to form one cluster. DBSCAN clusters points not only into several clusters but also determines the points that could not be assined to any cluster. Which is also very useful as the data used in this research will be quite noisy.