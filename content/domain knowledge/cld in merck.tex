There are many different proteins that can be produced using such technologies, for example, vaccines, hormones, sugars and etc., however this research is dedicated to the production of monoclonal antibodies (mAbs). 

CHOZN® Platform is a currently widely used product of Merk KgaA. CHOZN is a CHO mammalian cell expression system for the fast and easy selection and scale up of clones producing high levels of recombinant proteins [cite tech-bulletin]. The processes of developing the expression systems on this platform correspond to the general CLD process described in the previous subsection [put subsection number]. The scope of the project is to facilitate the prediction of stability of the expression system by inducing an \textit{in silico} predictions of productivity and stability rates on early steps on the CLD process.

In order to choose single best cells for further cloning one asseses several parameters like cell size, granularity, cell surface protein expression and etc. This can be done with Fluorescent Activated Cell Sorting (FACS) technology that allows to sort signle cells. (https://doi.org/10.1517/14712598.4.11.1821). Unfortunately fluorescence labeling is expensive and may ruin the cell due to its phototoxicity (https://doi.org/10.1371/journal.pone.0007497). 