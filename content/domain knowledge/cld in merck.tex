There are many different proteins that can be produced using such technologies, for example, vaccines, hormones, sugars etc., This research however is dedicated to the production of monoclonal antibodies (mAbs). 

CHOZN® platform is a currently widely used product of Merk KGaA. CHOZN® is a CHO mammalian cell expression system for fast and easy selection and growth of clones producing high levels of recombinant proteins (\cite{chozn}). The processes of developing expression systems on this platform correspond to the general CLD process described in the previous subsection \ref{section:cld-steps}. The scope of the project is to simplify the labour-intensive and time-consuming process of stability measurement of the expression system by inducing predictions of productivity and stability rates during early steps in the CLD process. 

After the transfection step there are several quantities that are measured in minipools in order to select the best ones. For example, cell size, its complexity, cell surface protein expression, endoplasmic reticulum (ER) mass, mitochondria mass, etc. For qualitative and quantitative characterization of cells, fluorescent labeling is used. It is a process of covalently binding fluorescent dyes to biomolecules such as nucleic acids or proteins, so that they can be visualized via fluorescence imaging (\cite{fluorescent_labeling}). A fluorophore is a chemical compound that can reemit light at a certain wavelength.These compounds are a critical tool in biology because they allow experimentators to capture particular components of a given cell in detail (\cite{DL_for_LS} p.113).

Unfortunately, fluorescence labeling is expensive, time-consuming and may kill the cell due to its phototoxicity (\cite{Fried_1982}, \cite{Patil_2018}, \cite{Progatzky_2013}). Additionally, \cite{Yeo_2017} found out that different selection markers affect the production stability of CHO cells. Other negative aspects  the manual staining approach are: there is a limited number of available fluorescent channels in microscopes; some fluorophores have a spectral overlap, hence there is a limited number of detectable markers (\cite{Perfetto_2004}); such labels can be inconsistent (\cite{Burry_2011}, \cite{Weigert_1970}), and depend a lot on reagent quality and require many hours of lab work. Toxicity, for instance, is a very dangerous factor, especially for medicine production as it may even affect the final product. Because of this there is a need for an approach of \textit{in silico} fluorescent labeling - computationally and without affecting the cell. 

For \textit{in silico} labeling, the input data is a differential interference contrast (DIC) microscopy. This is an optical microscopy technique used to enhance the contrast in unstained, transparent samples [cite wikipedia?]. This is a much cheaper image acquisition technique than a staining process, and it has much less variability as well (for example, no dependency on the dye or antibody quality). The research carried out in the scope of this thesis is dedicated to predicting fluorescence signal from the DIC imaging directly without the need of actual cell staining. The measurements needed for selecting the minipools can be calculated as usual, with the exception of using the predicted images instead.
