There are many different proteins that can be produced using such technologies, for example, vaccines, hormones, sugars and etc., however this research is dedicated to the production of monoclonal antibodies (mAbs). 

CHOZN® Platform is a currently widely used product of Merk KgaA. CHOZN is a CHO mammalian cell expression system for the fast and easy selection and scale up of clones producing high levels of recombinant proteins [cite tech-bulletin]. The processes of developing the expression systems on this platform correspond to the general CLD process described in the previous subsection [put subsection number]. The scope of the project is to simplify labour intensive and time-consuming process of stability determination of the expression system by inducing predictions of productivity and stability rates on early steps on the CLD process. 

After the transfection step there are several quantities that are measured in minipools in order to select the best ones. For example, cell size, its complexity, cell surface protein expression, Endoplasmic Reticulum (ER)mass, Mitochondria mass and etc. For qualitative or quantitative characterization of cells fluorescent labeling is used. It is the process of covalently binding fluorescent dyes to biomolecules such as nucleic acids or proteins so that they can be visualized by fluorescence imaging [cite https://www.nature.com/subjects/fluorescent-labelling]. A fluorophore is a chemical compound that can reemit light at a certain wavelength.These compounds are a critical tool in biology because they allow experimentalists to image particular components of a given cell in detail [cite O'reilly life sciences p113].

Unfortunately fluorescence labeling is expensive, time-consuming and may ruin the cell due to its phototoxicity [cite Fried et al., 1982; Patil et al., 2018; Progatzky et al., 2013)]. Additionally, Yeo et al. [cite Tihanyi] found out that different selection markers affect the production stability of CHO cells. Other negative aspects of manual staining appoach are the following: there is a limited number of available fluorescent channels in microscopes; some fluorophores have a spectral overlap, therefore there is a limited amout of detectable markers [cite Perfetto et al., 2004]; such labels can be inconsistent [cite Burry, 2011; Weigert et al., 1970), depend a lot on reagent quality and require many hours of lab work. Toxicity for instance is a very dangerous disadvantage especially for the medince production as it may affect event the final product. Therefore there exists a need for an approach of \textit{in silico} flurescent labeling - computationally and without intervening into the cell. 

For the \textit{in silico} labeling the inputs data is a differential interference contrast (DIC) microscopy, this is an optical microscopy technique used to enhance the contrast in unstained, transparent samples [cite wikipedia?]. This is a much cheaper image acquision technique that the staining process, and there is much less variability within it as well (no dependance on the dye or antibody quality for ex.). The research is dedicated towards predicting fluorescence signal from the DIC imaging directly without the need of cell staining. The measurements needed for selection of minipools cant hen be calculated as usual, but using the predicted images instead.
