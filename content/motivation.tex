Nowadays recombinant proteins are spreadly used in biomedical research, the production of medicines that are used in the variety of therapeutic neeeds like vaccines and antibodies. Therefore there is currently a great need for the recombinant proteins production in high volumetric amounts and of a good quality. That is why an optimization and improvement of the cell line development (CLD) as a process that is used for the production of recombinant proteins is extremely important.

Clone screening as the analysis of the cells for further selection of the most stable and productive clones is one of the CLD steps. Fluorescence microscopy provides data about the cell structure that allows to perform a better clone selection however it is not only also expensive, time-consuming but it is also toxic for the cells. Therefore automating fluorescence microscopy for the clone selection via convolutional neural networks  \textit{in silico} significantly simplifies the existing procedure of the clone selection, reduces photoxicity time and expenses needed for the analysis.

As the goal of this thesis is to provide a proof of concept on wether an \textit{in silico} approach for fluorescent labeling can substitute the manual cell dyeing and provide all the needed information that would be used for the further clone screening and selection That is why this research is oriented for the specific needs, pipelines and data used in Merck KgaA. In this research four UNet models (for 4 target proteins highlighting different cell organelles) were developed for automating the fluorescence cell imaging from the DIC imaging of CHO cells: nuclei, endoplasmic reticulum, green fluorescent protein and Golgi apparatus. Another important aim of this research that differentiates it from the similar studies like TODO cite LaChance 2020 and cite Christiansen 2018 is to not only provide deep learning models for the fluorescence predicions but also study their reliability and be able to detect the drift within the imaging that can happen quite easily do to the sensitivity of the microsope settings and well as the cell phenotypes, scaling and fixation procedures.

The upcoming sections of the thesis are structured as follows: Section 2 provides a review of the biological concepts needed to understanding the application on this research and well as the machine and deep learning concepts used for the data analysis; Section 3 provides an overview of the implementation and the results of the experimental \textit{in silico} fluorescence predictions; Section 4 reviews the stability of the deep learning models developed in the previous section and provides valuable insights on the information from their embeddings; Section 5 reviews the useful tools used for the delepment in Merck KgaA and Section 6 explores the possible future research questions that were rose from the current analysis and provides concluding remarks.
