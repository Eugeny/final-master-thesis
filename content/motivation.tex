Nowadays recombinant proteins are widespread in biomedical research and production of medicines like vaccines and antibodies that are used in the variety of therapeutic needs (\cite{Liu_2022}, \cite{Kim_2011}, \cite{Jayapal_2007}). Therefore there is currently a great need for high-volume and high-quality recombinant protein production. That is why the optimization and improvement of cell line development (CLD) as the main process used for the production of recombinant proteins is extremely important and relevant.

Clone screening is a step of CLD process in which cells are analyzed for further selection of the most stable and productive clones. Fluorescence microscopy used in this case provides data about the cell structure that enables better clone selection. However, it is not only expensive and time-consuming procedure, but is also toxic for the cells. Automating fluorescence microscopy for clone selection via convolutional neural networks \textit{in silico} significantly simplifies the existing procedure of clone selection, reducing phototoxicity, time and expenses needed for the analysis. Therefore, further search for improvement in the process of automation of microscopy fluorescence is an urgent task.

The goal of this thesis was to provide a proof of concept on whether an \textit{in silico} approach to fluorescent labeling can substitute manual cell staining. Proving successfull automatization would mean that all information needed further clone screening and selection already exists in DIC imaging and cell staining can be escaped. The research carried out in this work was aimed towards the specific needs, pipelines and data used at Merck KGaA. While the general goal of the project is to predict cell productivity and stability, this study provides not only a solution to escape manual cell staining, but also features from DIC imaging that can be used for productivity prediction directly. In this research four UNet models (for four target proteins highlighting different cell organelles) were developed for automating fluorescence cell staining based on DIC microscopy imaging of CHO cells. These targets are: nuclei, endoplasmic reticulum, Golgi apparatus and full cell surface. The aim of this study that differentiates it from the similar studies like \cite{Lachance_2020} and \cite{Christiansen_2018} was to not only provide deep learning models for the fluorescence predictions, but to study their reliability and to be able to detect drift during image acquisition. Image acquisition error can arise quite easily due to the sensitivity of the microscope settings as well as the cell phenotypes, scaling and fixation procedures.

This thesis is laid out as follows: Section 2 reviews the biological concepts needed to understand the application of this research, it also reviews machine and deep learning concepts used for data analysis; Section 3 provides an overview of the implementation and the results of the experimental \textit{in silico} fluorescence predictions; Section 4 shows stability of the deep learning models developed in the previous section and provides valuable insights on the information from their embeddings; Section 5 explores possible future research questions that arose from the current analysis and Section 6 provides concluding remarks and succinct recommendations.