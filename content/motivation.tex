Nowadays recombinant proteins are widely used in biomedical research and production of medicines that are used in the variety of therapeutic neeeds like vaccines and antibodies [TODO add references]. Therefore there is currently a great need for high-volume and high-quality recombinant protein production. That is why the optimization and improvement of cell line development (CLD) as a process in use for the production of recombinant proteins is extremely important.

Clone screening is a step of the CLD process in which cells are analyzed for further selection of the most stable and productive clones. Fluorescence microscopy provides data about the cell structure that enables better clone selection, however it is not only expensive and time-consuming, but also toxic for the cells. Automating fluorescence microscopy for clone selection via convolutional neural networks \textit{in silico} significantly simplifies the existing procedure of clone selection, reducing phototoxicity, time and expenses needed for the analysis.

The goal of this thesis is to provide a proof of concept on whether an \textit{in silico} approach to fluorescent labeling can substitute manual cell staining and provide all the needed information that would be used for further clone screening and selection. That is particularly why the research at hand is aimed towards the specific needs, pipelines and data used at Merck KgaA. In this research four UNet models (for four target proteins highlighting different cell organelles) were developed for automating fluorescence cell staining based on DIC microscopy imaging of CHO cells: nuclei, endoplasmic reticulum, [[green fluorescent protein]] and Golgi apparatus. Another important goal of this research that differentiates it from the similar studies like [TODO cite LaChance 2020 and cite Christiansen 2018] is to not only provide deep learning models for the fluorescence predictions but also study their reliability and be able to detect drift during image acquisition that can happen quite easily due to the sensitivity of the microscope settings as well as the cell phenotypes, scaling and fixation procedures.

This thesis is laid out as follows: Section 2 reviews the biological concepts needed to understand the application of this research, it also reviews machine and deep learning concepts used for data analysis; Section 3 provides an overview of the implementation and the results of the experimental \textit{in silico} fluorescence predictions; Section 4 shows stability of the deep learning models developed in the previous section and provides valuable insights on the information from their embeddings; Section 5 details the practical tools used for the development at Merck KgaA and Section 6 explores possible future research questions that arose from the current analysis and provides concluding remarks and succinct recommendations.