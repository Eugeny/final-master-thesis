Nowadays recombinant proteins are widespread in biomedical research and production of medicines like vaccines and antibodies that are used in the variety of therapeutic needs (\cite{Liu_2022}, \cite{Kim_2011}, \cite{Jayapal_2007}). Therefore there is currently a great need for high-volume and high-quality recombinant protein production. Optimization and improvement of cell line development (CLD) as the main process used for the production of recombinant proteins is therefore extremely important and relevant.

Clone screening is a step of CLD process in which cells are analyzed for further selection of the most stable and productive clones. This process includes differential interference contrast (DIC) microscopy of the cell and analysis of cell structure in terms of its organelles characterization. Imaging of cell organelles can be acquired via fluorescence microscopy after a staining procedure that allows to highlight an organelle of interest in fluorescence spectrum. However, it is not only expensive and time-consuming procedure, but is also toxic for the cells. Automating fluorescence microscopy via neural networks \textit{in silico} significantly simplifies the existing procedure of clone selection, reducing phototoxicity, time and expenses needed for the analysis. Additionally, microscopy image acquisition errors can arise quite easily due to the sensitivity of the microscope settings as well as the cell phenotypes, scaling and cell fixation procedures. Therefore, further search for improvements in the process of automation of microscopy fluorescence and testing its robustness towards different input corruptions are urgent tasks.

The goal of this thesis was to provide a proof of concept on whether an \textit{in silico} approach to fluorescent labeling can substitute manual cell staining. Successful automation of this process would prove that all information needed further clone screening and selection already exists in DIC imaging and cell staining can be escaped entirely. The research carried out in this work was aimed towards the specific needs, pipelines and data used at Merck KGaA. While the general goal of the project is to predict cell productivity and stability, this study provides not only a solution to escape manual cell staining, but also features from DIC imaging that can be used for productivity prediction directly. There are four cell targets of interest for this project that should be predicted based on DIC microscopy: nuclei, endoplasmic reticulum, Golgi apparatus and full cell surface. The aim of this study that differentiates it from the similar studies like \cite{Lachance_2020} and \cite{Christiansen_2018} was to not only provide deep learning models for the fluorescence predictions, but to study their reliability and to be able to detect drift during image acquisition.

This thesis is laid out as follows: Section 2 reviews the biological domain knowledge of CLD processes, machine and deep learning concepts used for image analysis; Section 3 provides an overview of the implementation and the results of \textit{in silico} fluorescence predictions; Section 4 explores stability of the deep learning models developed in the previous section and provides valuable insights on the information from their embeddings; Section 5 shows possible future research questions that arose from the current analysis and Section 6 provides concluding remarks and succinct recommendations.