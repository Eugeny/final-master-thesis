
As \cite{Lachance_2020} notice evaluation of models predicting flureoscence signal in terms of the PCC or MSE loss is not quite an objective approach. Even when one model has a smaller loss, it will not nessecarily perform better than another model. There are other more practical every-day metrics for this task, that have more value to the end-user rather than some abstract performance measurements typically used in computer vision. Therefore the evaluation of the model in this research is additionally checked in terms of the following practical metrics:
\begin{itemize}
    \item Number of nuclei / ER / Golgi / cells
    \item Area of nucleus / ER / Golgi / cell
    \item Total intensity of nuclei / ER / Golgi
    \item Mean intensity of nuclei / ER / Golgi
\end{itemize}

These quantities can be compareed with each other in a much more understandable way. On the contrary, when one recieves a precision value $P$ (let it be PCC loss in this example) from the model performance evaluation there is no way to appreciate how good this model is. Usually the value of $P$ can easily be increased by simply training on more data or using a better resolution microscopy, but this would increase the amount of lab work and expenses.

Highly practical metrics mentioned above would be calculated for each image in the test set and for corresponding ground truth images. This would give us two distributions: one is a distribution of predicted values and another is a distribution of ground truth values. Ideally, both of them should be the same distribution. In order to compare these distributions in this work three approaches are used. First, is to visually compare the violin plots by simply checking the form and range of the two. Second, to visualize a scatter plot, where two axes have values of a metric for each image from both distributions. And third, to compare them quantitevely by correlation coefficients, or more specifically, with the help of Spearman rank coefficient and Pearson correlation coefficient. Such comparisons are performed for each of the four metrics and for each organelle.

In each subsection dedicated to each organelle these metrics are presented under the according "Downstream metrics" section.