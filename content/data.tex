There were 4 targets studied in this research: nuclei, endoplaspic reticulum, Golgi apparatus and full cell. First 3 targets provide many insights on the state of cell based on their quaintity, size and the intensity of fluorescence signal that they can produce by binding with the protein, the latter one is usefull in general for locating cells and calculting their area. There are of course other possible targets that could be used during the selection step of CLD, but the goal of this particular study was to provide that the proof of concept on wether the netwook could potentially sunstiture manual fluorescence staining in general. The hypothesis used here is that if the model can successully predict fluorescence signal from some of the targets, then the research could be exended to other cell organelles as well.

The imaging data comes always in pairs (DIC + fluorescence imaging) and has the following structure:


\dirtree{%
    .1 dataset0.
        .2 DIC.
            .3 {\{CHOZN, PHX, H19\}\_XY001.tif}.
            .3 {$\cdots$}.
            .3 {\{CHOZN, PHX, H19\}\_XY100.tif}.
        .2 \{DAPI\_nuclei, FITC, FITC\_er, CY5\_golgi\}.
            .3 {\{CHOZN, PHX, H19\}\_XY001.tif}.
            .3 {$\cdots$}.
            .3 {\{CHOZN, PHX, H19\}\_XY100.tif}.
    .1 {$\cdots$}.
    .1 datasetN.
        .2 {$\cdots$}.
  }

Here $\text{dataset}\{0, \cdots N\}$ is one 96-well plane. There are only several datasets for each of organelle and each of them contains 100 images on random locations within this plate. Each dataset includes the 2 subfolders - DIC and a fluorescence subfolder. The name of the latter one corresponds to the name of the fluorescent binding molecule as they are different for each target within the cell. Almost all plates contain pairs of images form one fluorescent staining target only. Each filename within the subfolder includes its index number between $\{1, \cdots 100\}$ and the phenotype of the cell. "Cellular phenotype is the conglomerate of multiple cellular processes involving gene and protein expression that result in the elaboration of a cell's particular morphology and function" [TODO cite Jai-Yoon Sul 2009]. There are only 3 different phenotypes present in the dataset: CHOZN, PHX and H19. The latter however one is present only for full cell target.

\begin{table}[H]
    \centering
    \caption{Available data for each fo the organelles}
        \begin{adjustbox}{width=0.7\textwidth}
            \begin{tabular}{|c||c|c|c|c|}\hline
                &Total images
                &Training crops
                &Validation crops
                &Test crops
                \\\hline\hline
                Nuclei & $595$ & $27,264$ & $3,008$ & $7,616$\\\hline
                Actin &$400$ & $18,432$ & $2,048$ &$5,120$\\\hline
                Golgi & $761$ & $23,036$ & $2,336$ & $6,347$\\\hline
                H19 & $400$ & $18,432$ & $2,048$ & $5,120$ \\\hline
                Nucleolei &?&?&?&?\\\hline
            \end{tabular}
        \end{adjustbox}
    \label{table:data}
\end{table}

For the training all images are cropped into smaller crops of size $256 \times 256$ and split between training, test and validation sets with crops from one image being present in only one set. However it is still possible that the cells themselves might be mixed up between the sets, because the of the randomness of the microscopy focusing system (see subsection \ref{par:crops-combination}). The summary of the total number of crops in each set is presened in Table \ref{table:data}.

