Training the model on the ER stain images produced very successful results that can be used as a replacement for manual staining with \textit{in silico} labeling. Overfit of the model has been encountered and several experiments were performed to overcome it, the best checkpoint was selected and evaluated.

The segmentation of ER for further biological metrics evaluation was proposed. The problem of overlap between cells being present was discovered, which causes the segmented regions to merge together. Subsequently, a better preprocessing algorithm was proposed. This algorithm helps to avoid the artifacts appearing from a local thresholding.

The combination of ER with nuclei predictions was visualized and analysed, it was confirmed that the models produce realistic results confirming biological definition of their mutual arrangement within the cell. However, it is recommended to test the models using the simultaneous staining of both organelles to get quantitative metrics of the accuracy.