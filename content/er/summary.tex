Training the model on the ER stain images produced very successfull results that can be used as as replacement for manual staining with \textit{in silico} labeling. Overfit of the model has been encountered and several experiments were performed to overcome it, the best checkpoint was selected and evaluated.

The segmentation of ER for further downstream metrics evaluation was proposed. The problem of presence of the overlap between the cells causing the segmented regions to merge together is discovered and a better preprocessing algorithm was proposed. This algorithm helps to avoid the artifacts appearing from a local thresholding.

The combination of ER with nuclei predictions was visualized and analysed, it was confirmed that the models produce realistic results confirming biological definition of their mutual arrangement within the cell. However, it is recommended to test the models using the simultaneous staning of both organelles to get a quantative metrics of the accuracy.