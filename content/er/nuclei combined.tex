Now having two models for the predictions of two organelles: nuclei and ER, it is interesting to visualize the predictions together (see Figure \ref{fig:er-combined}). It is clear from the image that ER indeed is located around the nucleus. Unfortunately, the cells were stained only for separate oragnelles and different fluorescent targets do not intersect, meaning there were no images with both nucleus and ER stained at the same time. Further research can be applied here to in order to evaluate the models even better, measuring the intersection error between the two for instance.
\begin{figure}[htb]
    \centering
    \setkeys{Gin}{width=\linewidth}
    \centering
        \begin{tabularx}{\textwidth}{YYYY}
            \textbf{Ground truth} &
            \textbf{Prediction} &
            \textbf{Prediction + nuclei} \\
            \includegraphics{bilder/ER/gt.jpg} & \includegraphics{bilder/ER/er.jpg} &
            \includegraphics{bilder/ER/gt_nuclei.jpg}
        \end{tabularx}
    \caption{Combination of ER with nuclei}
    \label{fig:er-combined}
\end{figure}