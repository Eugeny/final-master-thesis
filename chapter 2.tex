\section{Background}
\subsection{Biology}
CLD
Cell line
fluorescence
DIC
vector
CHO cell
proteins
transfection

Explain CLD process
Its need for producing proteins
Explain how cells are chosen and that it long
Explain the role of fluorescence
Explaing its pro and cons

Chinese hamster ovary (CHO) cells are ones of the most popular mammalian cells used for protein production. (doi:10.1016/B978-0-08-100623-8.00007-4) 

First step of cell line development (CLD) is the
introduction of the gene of interest (GOI) to CHO cells, this process is called a transfection. It is important to transfect a GOI (or a vector) into an optimal site of genome to secure a high protein expression over time during protein production, but transfection mostly happens into a random place. In case if the gene was transfected in the inactive site of genome (and the majority of genome is not transriptionally active) the cell likely will not express this gene. (doi:10.1016/B978-0-08-100623-8.00007-4) (doi:10.1016/j.coche.2018.08.002) However nowadays it is also possible to insert a GOI into a specific genomic location via CRISPR/Cas9 technology. 

The second step is to select cell pools that have successful and stable gene integrations. The reasond why not all of them are suitable for cloning is the following: during the transfection only 80\% of cells recieve the vector of GOI (doi:10.1016/B978-0-08-100623-8.00007-4), only the small percent of them actually integrate a vector into the genome (as mentioned above) and only a fraction of those cells are able to stabily produce a protein. (Reference needed). This could be done with bulk sorting algorithm. (doi:10.1016/B978-0-08-100623-8.00007-4)

The third step in CLD is to clone the cells. The chosed stable pools of cells are phenotypically and genetically diverse - have different growth rates, metabolic profile and etc. This is not ideal for industrial production ([25] here doi:10.1016/B978-0-08-100623-8.00007-4). Fluorescent Activated Cell Sorting (FACS) technology is used to choose single best cells for further cloning based on several parameters like cell size, granularity, cell surface protein expression and etc. (https://doi.org/10.1517/14712598.4.11.1821). Unfortunately fluorescence labeling is expensive and may ruin the cell, so there is a need to label cell parts in silico without intervening into the cell with the chemicals that are used for the fluorescence labeling. There is also a limited number of available fluorescent channels in microscopes, phototoxicity. (https://doi.org/10.1371/journal.pone.0007497). Such labels can also be inconsistent, depend a lot on reagent quality, and reqiere human lab work hours.

The forth step is cloning - when the whole population is derived just from one chosed cell, it strongly reduces phenotipical and genetical divertsity. 

Once the cells are cloned, the next step is to charaterize the them based on the following criteria: cell size, growth rate, protein quality, titer, metabolities and etc. 

After 90 days of observations one can determine which cells are stable and therefore suitable for production. This is the last step of CLD process and consumes a lot of time and maintaining of the cells. Predicting the stability of the cells directly from DIC images would reduce this time significantly allowing to escape this process completely.

\subsection{ML}
Background on Unet and ML in general