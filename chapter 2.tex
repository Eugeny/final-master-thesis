\section{Background}
\subsection{Biology}
CLD
Cell line
fluorescence
DIC
vector
CHO cell
proteins
transfection

Explain CLD process
Its need for producing proteins
Explain how cells are chosen and that it long
Explain the role of fluorescence
Explaing its pro and cons

Cell line development (CLD) is the process by which the cellular machinery is co-opted to manufacture therapeutic biologics or other proteins of interest. One can use different expression systems for cell line development: bacterial, plant-based, yeast, mammalian. (copy paste from https://www.beckman.de/resources/product-applications/lead-optimization/cell-line-development) Chinese hamster ovary (CHO) cells are the most popular mammalian cells used for protein production. (doi:10.1016/B978-0-08-100623-8.00007-4) 

First step of CLD is the
introduction of the gene of interest (GOI or a DNA vector) to CHO cells. This process is called a transfection. It is important to transfect a GOI into an optimal site of genome to secure a high protein expression over time during protein production, however pratically transfection mostly results in a vector being inserted into a random site within a host cell genome. In case the gene was transfected in the inactive site of genome (and the majority of genome is not transriptionally active) the cell will likely not express the gene. (doi:10.1016/B978-0-08-100623-8.00007-4) (doi:10.1016/j.coche.2018.08.002)

The second step is the selection of cell pools that have successful and stable gene integrations. The reasond why not all of them are suitable for cloning is the following: during the transfection only 80\% of cells recieve the vector of GOI (doi:10.1016/B978-0-08-100623-8.00007-4), only the small percent of which actually integrate a vector into the genome and, as mentioned above, only a fraction of those cells are able to stabily express a protein. (Reference needed). Such selection could be done with bulk sorting algorithm. (doi:10.1016/B978-0-08-100623-8.00007-4)

The third step in CLD is to clone the cells. The chosen stable pools of cells are phenotypically and genetically diverse - meaning they have different growth rates, metabolic profile and etc. This is not ideal for industrial production - all the cells used for protein production should be derived from the same clone ([25] here doi:10.1016/B978-0-08-100623-8.00007-4). In order to choose single best cells for further cloning one asseses several parameters like cell size, granularity, cell surface protein expression and etc. This can be done with Fluorescent Activated Cell Sorting (FACS) technology. (https://doi.org/10.1517/14712598.4.11.1821). Unfortunately fluorescence labeling is expensive and may ruin the cell due to its phototoxicity (https://doi.org/10.1371/journal.pone.0007497). There is a limited number of available fluorescent channels in microscopes as well as such labels can also be inconsistent, depend a lot on reagent quality, and require many hours of lab work. Therefore there exists a need for flurescent labeling in silico - without intervening into the cell. 

Once the cells are cloned, phenotypical and genetical heterogeneity is reduced, the next step is to charaterize clonally-derived cells based on the following criteria: cell size, growth rate, protein quality, titer, metabolities and etc. With this one can estimate clones productivity and titer. Such observations may take up to 90 days after which one can determine which cells are stable and therefore suitable for production. This is the last step of CLD process and consumes a lot of time and maintaining costs for feeding and cloning the cells. Predicting the stability of the cells directly from DIC images would reduce this time significantly allowing to escape this process completely.

\subsection{ML}
Background on Unet and ML in general