%%% Die folgende Zeile nicht ändern!
\section*{\ifthenelse{\equal{\sprache}{deutsch}}{Abstract}{Abstract}}
%%% Zusammenfassung:
%Cell line development is an expensive and time-consuming process, however that is the most modern approach for producing the proteins needed in various pharmaceuticals.

Nowadays there is a great need for high-volume and high-quality recombinant protein production, yet cell line development (CLD) processes used for this are expensive and time-consuming. The main objective of this thesis was to provide a proof of concept in application of a novel in-silico fluorescence labeling approach to simplify the existing procedures of clone selection step in CLD. As a result, a solution to reduce phototoxicity as well as time and expenses needed for cell analysis was offered by making staining of cells completely redundant. The experiments were carried out with the use of state-of-the-art deep learning models to predict four fluorescent targets in CHO cells based on their DIC imaging only. For their better evaluation more practical biological metrics were proposed. The reliability of the models was studied and a possibility to detect drift within the data was provided via statistical two-sample hypothesis testing.This research showed that manual staining process of three cell targets at Merck KGaA can be in this case fully substituted by \textit{in silico} fluorescence labeling models that were developed here. In case of Golgi apparatus target, it was recommended to acquire more data with a better signal-to-noise ratio. The results of this research showed successful development of data drift detection mechanisms that can alarm the end-user in case of wrong microscopy data acquisition settings. The results are of direct practical relevance to subsequent productivity predictions used for significant acceleration the CLD process.

\textbf{Key words:} \textit{in-silico} fluorescence labeling, microscopy, deep learning, computer vision, CHO cells, drift detection.