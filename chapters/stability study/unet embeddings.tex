\subsection{UNET embeddings study}
    Similarly to studying autoencoder embeddings that represent a high-dimentional input in lower dimentional space, one can study UNet embeddings. However, one should keep in mind that the dimention of embeddings in UNet case is not smaller than the input and often is larger (see the UNet architecture in Figure \ref{fig:unet}). The goal of the UNet in contrast to an autoencoder is not to compress the input, but to extract useful features that are helpful for high-resolution segmentation. UNet embeddings do not contain rich full image semantics in them as embeddings of an autoencoder do. UNet compresses the spatial dimention of the input but it also gradually increases the number of filters which can capture of information need for segmentation. As it has been proven in section \ref{section:nuclei-predictions} having more filters only helps to get better predictions, therefore there is no need for a UNet to have low-dimentional embeddings. Nevertheless, it is still interesting to see if the embeddings do contain any interesting information about the input that one could use. There were two hypothesis in question: the first one is whether embeddings of a trained UNet form any kind of clusters based on cells phenotype. And a second one is whether embeddings of corrupted images can be clustered together further away from not corrupted ones. If the latter hypothesis would hold, one could alarm the end-user about the outliers in the dataset based on their distance from both of the clusters. 
    \subsubsection{Application of various dimentionality reduction methods}
        It is important that any image is fed into the network crop by crop, meaning that for each crop there is a separate embedding. In this section crops embeddings were not combined in any way together and were analysed separately.

The UNet embedding has a size of $16 \times 16 \times 256$ and can be flattened into a $655536$-dimensional vector. In order to comprehend the embeddings for us as humans, a dimensionality reduction algorithm has to be applied. One option would be to compress a vector to 2D or 3D representation, which is easily comprehendable by humans.
\begin{figure}[htb]
	\includegraphics[width=\linewidth]{bilder/unet-embeddings/umap-pca-embeddings.png}
	\caption[Visualization of UNet embeddings in 2D space]%
	{Visualization of UNet embeddings in 2D space. (a) PCA, (b) UMAP, (c) combination of PCA and UMAP with 10 and (d) 200 components. First row differentiates between two cell phenotypes: CHOZN and PHX, whereas the second row differentiates between uncorrupted crops and crops corrupted using artificial defocus blur of severity level $4$.}\label{fig:umap-pca-embeddings}
\end{figure}

In this case a two-dimensional representation was chosen. With the help of PCA, UMAP and their combination embeddings were projected into a 2D space. In Figure \ref{fig:umap-pca-embeddings} we can see kernel density estimate (KDE) plots, that were created based on scatter plots, where each dot represents a projected UNet embedding of a crop. Both research questions are addressed here: clustering based on phenotypes (CHOZN or PHX) and clustering based on input corruptions (defocus blur of severity level $4$). It is crucial here that corrupted data was not used in training of any of dimensionality reduction method. The goal was to use only the data available in training dataset, find the transformation of high-dimensional data into a lower-dimensional space and apply it to new samples. That is also why methods like t-sne (\cite{t-sne}) cannot be used here, because the transformation that t-sne learns cannot be applied to new samples. 

From Figure \ref{fig:umap-pca-embeddings} it becomes evident that there is no clustering based on the phenotype. On the one hand, this means that it is not possible to detect phenotype based on the UNet embedding. But on the other hand, this also implies that PHX or CHOZN phenotypes do not influence the predictions so much supports previous conclusion from Section \ref{section:generalizability-across-phenotypes} that the model generalises well across them. Regarding the clustering based on the artificial corruption it seems that embeddings of corrupted samples tend to clump more in groups, occupying one specific area of the embeddings space. It is also clear that the combination of UMAP with previously applied PCA works better with the increasing amount of components in PCA: dots in (d) seem to form a better cluster than dots in (c). However, it can still not be taken intuitively from this figure how many non-corrupted dots are hidden behind the cluster of the green dots --- meaning whether non-corrupted crops cluster intersects severely with a corrupted one. In order to visualize this better, one can use a kernel density estimate (KDE) plot presented in Figure \ref{fig:kde}. Additionally, it is clear that pure UMAP is not the best approach for the extreme number of dimensions as with the one in this case.

%\begin{figure}[htb]
%	\begin{center}
%		\includegraphics[width=0.6\linewidth]{bilder/unet-embeddings/kde.png}
%	\caption{KDE plot of UMAP applied after PCA with 200 components}\label{fig:kde}
%	\end{center}
%\end{figure}

A cluster of corrupted images is clearly present here, however it also intersects with many non-corrupted crops. The quantitative evaluation of how this cluster is separable from the rest of the points is provided in section \ref{section:clustering-on-unet-embeddings}. Although one can already state that there is a clear opportunity to differentiate between corrupted and not corrupted images, the accuracy cannot be high due to the clusters being not well separable. For further research it is suggested to additionally check whether clusters form into a high-dimensional space before projecting them into a 2D space. 

\paragraph{Clustering with PaCMAP}
\label{section:clustering-on-unet-embeddings}
\begin{figure}[htb]
	\begin{center}
		\includegraphics[width=\linewidth]{bilder/unet-embeddings/PacMAP.png}
		\caption{Clustering of UNet Embeddings}\label{fig:unet-clustering}
	\end{center}
\end{figure}

\begin{figure}[htb]
	\begin{center}
		\includegraphics[width=0.6\linewidth]{bilder/unet-embeddings/db-levels.png}
		\caption{Clustering for different severities levels}\label{fig:unet-clustering-sev-levels}
	\end{center}
\end{figure}

TABLE with F1-score:
0.76 VS 0.64

    \subsubsection{Autoencoder embeddings as an alternative}
        Since UNet embeddings seem to not exhibit any exceptional results in termns of clustering, it was decided to train a vanilla autoencoder directly on image crops. Since autoencoder's embeddings contain dense semantic information of the input they might provide more insights for clustering hypotheses mentioned before. Figure \ref{fig:ae-training} presents the architecture of two convolutional autoencoders used for these experiments. One compresses $256 \times 256$ input crops into embeddings vector of size $3528$ and another one compresses them into a vector of a smaller size $200$. Both autoencoders were trained using MSE loss. The results of their convergence are presented in Figure \ref{fig:ae-training} on the right.

\begin{figure}[H]
	\begin{center}
		\includegraphics[width=\linewidth]{bilder/ae-embeddings/training-architectures.png}
		\caption{Architectures of two autoencoders and their training convergence}\label{fig:ae-training}
	\end{center}
\end{figure}

An autoencoder with embeddings of bigger size was able to achieve a lower loss as well as samples reconstructed from it were of a better quality (see Figure \ref{fig:ae-samples}). Clearly samples reconstruction will not have a high resolution as there are no skip-connections in this architecture. However, this is also not needed, the main goal here would be to find out whether autoencoder embeddings provide any insights on the data.
\begin{figure}[H]
	\begin{center}
		\includegraphics[width=0.5\linewidth]{bilder/ae-embeddings/ae-samples.png}
		\caption{Samples drawn from trained autoencoders}
		\label{fig:ae-samples}
	\end{center}
\end{figure}

Since an autoencoder with bigger embeddings size seems to be able to reconstruct crops much better we have proceeded with its achtitecture. Embeddings were projected into a 2-dimensional space using first PCA with 10 components and then applying UMAP on PCA's projections. The results of such projection are presented in Figure \ref{fig:ae-pca-umap-clustered}. Two clearly defined clusters appear: left plot presents projections from an earlier epoch, the right one --- from a later one. Embeddings separate gradually into two clusters throughout the training.

\begin{figure}[htb]
	\begin{center}
		\includegraphics[width=0.8\linewidth]{bilder/ae-embeddings/pca-umap-clusters.png}
		\caption{Autoencoder embeddings after applying PCA with 10 components and UMAP afterwards. Earlier epoch VS later epoch.}\label{fig:ae-pca-umap-clustered}
	\end{center}
\end{figure}

However, these two clusters are based neither on cell phenotype nor on input corruption. All points of both phenotype as well as corruptions seem to be equally mixed between two custers. By looking at the images correponding to each of the clusters it soon became clear that the main difference between them is their brightness level. To prove this theory distirbutions of average image intensity of images in both clusters are presented in Figure \ref{fig:ae-brighter-darker}. From violin plots it becomes clear that distribution of the crops on the left has a much lower brightness level than distribution of the crops on the right.
%\begin{figure}[htb]
%	\begin{center}
%		\includegraphics[width=0.5\linewidth]{bilder/ae-embeddings/pacmap.png}
%		\caption{PacMAP does not provide information on the coruption}\label{fig:ae-pacmap}
%	\end{center}
%\end{figure}

\begin{figure}[H]
	\begin{center}
		\includegraphics[width=0.5\linewidth]{bilder/ae-embeddings/brighter-darker.png}
		\caption{What do two UMAP clusters represent}
		\label{fig:ae-brighter-darker}
	\end{center}
\end{figure}

Since an autoencoder picks up on brightness difference within the crops, it is worth trying to normalize crops brightness across all dataset first. Nevertheless, it is not a trivial task as images have different cell density in them. That is why some images that contain primarily background pixel will always be darker than the ones that contain enough of foreground. We suggest to filter the crops based on amount of cells critea (which can be done using GFP model that can detect cells present in DIC) and normalize them afterwards. Retraining autoencoder on new training data might provide more insights when difference in brightness will be gone.

It is also clear why autoencoder embeddings do not provide any clustering for corrupted crops. Corruption severities neither really change the image semantics nor they are significantly different visually (see defocus blur in \ref{fig:artificial-corruptions}). Therefore they do not alter the ability of an autoencoder to restore input correctly. In contrast, UNet's fluorescence predictions do suffer significantly for severy corruption levels, its predictions strongly changes --- outline of the organelle becomes more blurry, additional shine appears in fluprescence prediction. These changes happen not only during the decoding part, but they also might bring unsual values in the embedding representation. Therefore when UNet embeddings have more information on the "trustworthiness" of predictions. That is why when defocus corruptions are used as training augmentations, drift detection for the model trained with these corruptions stops alarming about the drift, although it did for the model, which did not have these augmentations present TODO add reference. This happens simply because models predictions degrade and start looking different, which triggers a "drift alarm", with the imroved predictions, drift alarm would not be triggered even when using the same data.
    \clearpage
    \subsubsection{Clustering of PacMAP embeddings}
        \paragraph{Clustering on UNet embeddings}
        \label{section:clustering-on-unet-embeddings}
        \begin{figure}[htb]
	\begin{center}
		\includegraphics[width=\linewidth]{bilder/unet-embeddings/PacMAP.png}
		\caption{Clustering of UNet Embeddings}\label{fig:unet-clustering}
	\end{center}
\end{figure}

\begin{figure}[htb]
	\begin{center}
		\includegraphics[width=0.6\linewidth]{bilder/unet-embeddings/db-levels.png}
		\caption{Clustering for different severities levels}\label{fig:unet-clustering-sev-levels}
	\end{center}
\end{figure}

TABLE with F1-score:
0.76 VS 0.64