\subsection{Drift detection}
    \subsubsection{A need to detect drift}
    \subsubsection{Maximum mean discrepancy for drift detection}
        \begin{figure}[H]
	\begin{center}
		\includegraphics[width=0.8\linewidth]{bilder/drift-detection/fn-rate.jpg}
		\caption{False negatives rate for drift detection on artificial corruptions}\label{fig:fn-rate}
	\end{center}
\end{figure}

    \subsubsection{Online version of MMD algorithm}
            \begin{figure}[H]
	\begin{center}
		\includegraphics[width=0.6\linewidth]{bilder/drift-detection/online.png}
		\caption{Expected runtime (ERT) for corrupted and in-distribution data}\label{fig:online-ert}
	\end{center}
\end{figure}

\begin{table}[H]
    \centering
    \caption{Test window size influence on separability}
        \begin{adjustbox}{width=0.6\textwidth}
            \begin{tabular}{|l||*{5}{c|}}\hline
                \makebox{W}
                &\makebox[3em]{2}
                &\makebox[3em]{5}
                &\makebox[3em]{10}
                &\makebox[3em]{15}
                &\makebox[3em]{20}
                \\\hline\hline
                Auc-Roc &0.85&0.92&0.98&0.90&0.88\\\hline
            \end{tabular}
        \end{adjustbox}
\end{table}

\begin{table}[H]
    \centering
    \caption{ERT influence on separability}
        \begin{adjustbox}{width=0.5\textwidth}
            \begin{tabular}{|l||*{4}{c|}}\hline
                \makebox{W}
                &\makebox[3em]{32}
                &\makebox[3em]{64}
                &\makebox[3em]{128}
                &\makebox[3em]{256}
                \\\hline\hline
                Auc-Roc &0.90&0.95&0.98&0.98\\\hline
            \end{tabular}
        \end{adjustbox}
\end{table}
            \paragraph{Not fixed cells imaging as corrupted input}
                In section \ref{section:gfp} the difference between fixed and not fixed cells was mentioned. Visual analysis of model's predictions for not fixed cells after training it on fixed ones has shown that the model was not able to generalize well on them. That is why it would be important to alarm the end-user to not rely on predictions when such situation occurs. In this case an online drift detector trained using not corrupted data used for ER training first and tested on not fixed ER cells. The results of this test are shown in Figure \ref{fig:online-drift-not-fixed}.
                \begin{figure}[H]
                    \begin{center}
                        \includegraphics[width=0.5\linewidth]{bilder/drift-detection/online-fixed-vs-not-fixed.jpg}
                        \caption{Online drift detection of not fixated cells}\label{fig:online-drift-not-fixed}
                    \end{center}
                \end{figure}
                The ERTs for corrupted data (left) are lower from ERT for true input. ROC-AUC score for the separability is $0.91$ and the best threshold is $6$. However not corrupted data (fixed cells) mostly have ert of $7$, whereas corrupted data (not fixed cells) have an ert of $4$. Both classes have ERTs that are very close to the threshold.