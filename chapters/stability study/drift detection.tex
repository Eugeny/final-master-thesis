\subsection{Drift detection}
    Following the development phase, when the model training is finished, the model will be moved into deployment or production, where it is supposed to maintain an expected quality of predictions. However input data is not always a stable source of input. One should constantly maintain quality of predictions and do regular check-ups for outliers as well as to alert the end user about a drift in input data. Drift detection happens on raw data in absence of the ground truth labels and serves as a signal that the input data differs a lot from the data used for training, meaning that predictions became unreliable.

There is a significant difference between distinguishing drift of the whole source of data in comparison to detecting single outliers. In drift detection, one looks at the whole new input data as a distribution and checks if there is a significant shift in comparison to the data used during training.

There are two possible reactions after the drift is detected: alert the user that predictions became unreliable, and and therefore the expansion of the dataset should be considered by adding more labeled data from a newly drifted distribution in training, or applying some different logic on the model outputs. When an outlier is detected, a model might request human assistance for some particular input, because this input is too unfamiliar to the model and possibly it will not return good predictions on this one.

The goal of outlier detection is to decide on the single instances whether or not it is different from training data or unusual in one way or another. Outliers might appear in both training and predictions datasets.

Data drift and outlier detection can co-exist. It might be that the input has drifted, but there are no outliers, it might be that there are a lot of outliers, but the data was not drifted. (\cite{samuylova_2021}).

The key observation here is that the drift detector should be robust to outliers. The system should not send an alert as soon as it sees a suspicious sample due to the fact that outliers might be present in the original data distribution as well. But the alert should happen when there are many such samples. To compare original training data distribution with the new one from inputs different statistical tests like Kolmogorov-Smirnov, Chi-squared and others can used.

The need for maintaining drift detection or outlier detection depends on the cost of errors occurring. If the cost of a single error is too high, one should use an outlier detection, but when one needs a test to decide when to label new data - drift detection would be a better approach.

In summary, the drift detection is needed only when the meaningful shifts of the input data distribution from the training distribution need to be detected, whereas the outlier detector aims at finding unusual single instances in inputs. Here this is exactly the case, we train models assuming the correct setup of microscopy image acquisition, however changes in exposure, illumination, cell fixation procedure might alter DIC imaging. In this case the user has to be informed about it and choose afterwards whether more data should be added to the training set or whether the mode's predictions should not be used.

    \subsubsection{Drift detection experiments}
        Drift detection of corrupted samples for the problem of the thesis has been performed using \textit{alibi-detect} open source python library, that focuses specifically on outlier, adversarial and drift detection algorithms (\cite{alibi-detect}). This library implements statistical hypothesis testing algorithms for detecting drifts in data. 

It works the following way, before observing some data, one can specify null-hypothesis $H_0$ and alternative hypothesis $H_1$ about generating process behind the data (its distribution for example) and also specify the test statistics $S(X)$ that are expected to be small under hypothesis $H_0$ and large under hypothesis $H_1$. Then during the observation of the new data test statistic value $S(X)$ is computed along with a probability $p = P(S(X)|H_0)$ which is called p-value. P-value is a probability that such an extreme value of test statistic could have been observed under the null-hypothesis. If this probability is below the established threshold $t$, then one can assume that data is drifted. If p-value is low, null-hypothesis will be refused. 

A drift detection algorithm was trained on UNet embeddings of not corrupted train data. For this experiment $10,000$ embeddings were used of CHZN and PHX phenotypes from the nucleus dataset. It is important that the crops were chosen in such a way that at least several cells present are present there. After splitting images into the crops many of them contain a primarily background with few cells present. By filtering out these crops one can make sure that drift detection model actually learns on the important foreground signal from the cells rather than on the predictions of the background. After the drift detection algorithm was trained, it was tested on the dataset, that was not seen by a model beforehand (test dataset). Test dataset consists of $119$ images, where from each image $5$ random crops were chosen. The crops for each image were chosen again in the same way as fro training by only choosing the ones that have enough cells present in them. Since images have a high resolution, one can assume that one image itself represents a new input distribution, where crops taken from this image are its samples. Therefore we can detect whether one specific image has drifted or not feeding the crops from it into a drift detection algorithm. First, the algorithm was tested on not drifted data by using a test set of nucleus dataset. Out of $119$ images $8$ were recognized as drifted ones. This means that the algorithm's false positive is approximately $\frac{8}{119} \approx 0.063$.

Below the results of the trained drift detector for two datasets are presented: same test data with artificial corruptions applied to it and on data with real microscopy corruptions. 

\textbf{Artificial corruptions}

Figure \ref{fig:fn-rate} presents the results of drift detection for all artificial corruptions, more specifically the algorithm's false negative rate.
\begin{figure}[H]
	\begin{center}
		\includegraphics[width=\linewidth]{bilder/drift-detection/fn-rate.jpg}
		\caption[False negatives rate for drift detection on artificial corruptions]%
		{False negatives rate for drift detection on artificial corruptions- Long Description}{}\label{fig:fn-rate}
	\end{center}
\end{figure}
One can see that the lower the severity of a corruption is, the higher the false negative rate becomes. When the corruption severity level is low the predictions remain to have a high quality (see Figure \ref{fig:artificial-corruptions}), therefore an end user can still rely on the UNet. However, the stronger the corruption is, the stronger fluorescence prediction degenerates and as a result a drift detector alerts a user to the presence of drift. Drift detector is more sensitive towards contrast changes rather then towards defocus blur changes. It is the most sensitive towards brightness corruption.

\textbf{Real corruptions}

Types of real corruptions tested here are described in more details in Section \ref{section:real-corruptions}. Two phenotypes are present there: PHX (was also present in the training dataset) and 2e3 (was not present in any of the datasets before). Since these two subsets of data look very much alike drift detection results on both of them will be combined. In Table \ref{table:real-corruptions-dd} the results of the drift detection algorithm are presented. Additionally, for few samples of not drifted data that were also included in this dataset, namely 2 images of the correct focus distance Z and 4 images of the correct exposure time (30ms), the detector falsely alerted $0$ and $1$ samples correspondingly. The results confirm that the detector is trained well enough to be able to detect drift to some extent, however not all drifted images will be noticed by it. In order to state how exactly accurate it for real corruptions is much more data would be needed. Assuming that uncorrupted test dataset is representative enough, the low false positive rate is expected (around $0.063$). That is why one can presumably rely on the detector to alert the user about wrong exposures or focus corruptions. Regarding wrong fixation time of the cells, on the one hand it seems that this corruption has somewhat low detection rate, on the other one model's predictions are actually of quite a good quality. Therefore it might be the model is generally quite stable towards the prolonged fixation time.

\begin{table}[htb]
    \centering
    \caption{Drift detection for real microscopy corruptions}
        \begin{adjustbox}{width=\textwidth}
            \begin{tabular}{|c||c|c|c|c|c|}\hline
				
                &15 minute fixation
                &-5 Z
                &+5 Z
                &20 ms exposure
                &40 ms exposure
                \\\hline\hline
            	Detections & 3 & 2 & 3 & 2 & 2\\\hline
                Total images & 8 & 4 & 4 & 4 & 4\\\hline
            \end{tabular}
        \end{adjustbox}
    \label{table:real-corruptions-dd}
\end{table}
    \subsubsection{Online drift detection experiments}
            \begin{figure}[H]
	\begin{center}
		\includegraphics[width=0.6\linewidth]{bilder/drift-detection/online.png}
		\caption{Expected runtime (ERT) for corrupted and in-distribution data}\label{fig:online-ert}
	\end{center}
\end{figure}

\begin{table}[H]
    \centering
    \caption{Test window size influence on separability}
        \begin{adjustbox}{width=0.6\textwidth}
            \begin{tabular}{|l||*{5}{c|}}\hline
                \makebox{W}
                &\makebox[3em]{2}
                &\makebox[3em]{5}
                &\makebox[3em]{10}
                &\makebox[3em]{15}
                &\makebox[3em]{20}
                \\\hline\hline
                Auc-Roc &0.85&0.92&0.98&0.90&0.88\\\hline
            \end{tabular}
        \end{adjustbox}
\end{table}

\begin{table}[H]
    \centering
    \caption{ERT influence on separability}
        \begin{adjustbox}{width=0.5\textwidth}
            \begin{tabular}{|l||*{4}{c|}}\hline
                \makebox{W}
                &\makebox[3em]{32}
                &\makebox[3em]{64}
                &\makebox[3em]{128}
                &\makebox[3em]{256}
                \\\hline\hline
                Auc-Roc &0.90&0.95&0.98&0.98\\\hline
            \end{tabular}
        \end{adjustbox}
\end{table}
            \paragraph{Impact of cell fixation}
                In section \ref{section:gfp} the difference between fixed and not fixed cells was mentioned. Visual analysis of model's predictions for not fixed cells after training it on fixed ones has shown that the model was not able to generalize well on them. This is the reason why it would be important to alarm the end user to not rely on predictions when such a situation occurs. In this case an online drift detector trained using not corrupted data used for ER training first and tested on not fixed ER cells. The results of this test are shown in Figure \ref{fig:online-drift-not-fixed}.
                \begin{figure}[htb]
                    \begin{center}
                        \includegraphics[width=0.5\linewidth]{bilder/drift-detection/online-fixed-vs-not-fixed.png}
                        \caption{Online drift detection of not fixated cells}\label{fig:online-drift-not-fixed}
                    \end{center}
                \end{figure}
                The ERTs for corrupted data (left) are lower from ERT for true input. The ROC-AUC score for the separability is $0.91$ and the best threshold is $6$. However, not corrupted data (fixed cells) mostly have an ERT of $7$, whereas corrupted data (not fixed cells) have an ERT of $4$. Both classes have ERTs that are very close to the threshold, but are able to separate the classes well enough.

                Application of a usual drift detection algorithm with the use of ER model the false positive rate on not corrupted (fixed) cells was $0.075$. Whereas all fixed cells were recognized as drift.  