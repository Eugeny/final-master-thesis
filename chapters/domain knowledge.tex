\section{Domain knowledge}
% taken from here https://pubs.acs.org/doi/full/10.1021/acsphotonics.2c00599?casa_token=_MabQ6pGe48AAAAA%3A3cKiQjee69lw88NnkGzeH3OiTfHFd71Z4NjOJWBpdIUMNYMERNJ6mu9UpaTOYhZT7K8nlmxvJf7EqLHD

    \textit{In silico} fluorescence labeling approach has proven to be very promising as a substitute to the manual cell statining processes [TODO cite all the relevant references]. For example, the research of [TODO cite Christiansen 2018] proves successfull prediction of not only different cell stains for a variety of modalities and cell types, but it can also successfully determine cell viability. However the work is limited mainly to the transmitted light (TL) z-stack imaging, which means that the input to the network is comprised of 3D imaging, which is not the case in this research. [cite Ounkomol 2018] as well shows succeful predictions of several organelles in bright-field TL 3D imaging using 3D convolutional neural networks. There are two very promising studies by [cite Cheng 2021] and [cite LaChance 2020]. Even though the first one manages to reach a state-of-the art performance on label-free fluorescence reconstruction, it uses reflectance images from oblique dark-field illumination as the input, which is a more specific cell imaging acquisition technique. Whereas the second one provides an approach of using an easie imaging technique - DIC imaging as an input, which even shows great results with not high-resolution data. The latter one 
    
    All of the studies mentioned above as well as this thesis rely on the premise that input imaging type (DIC in this work) comprises enough information to predict fluorescence signal from it. Which is a reasonable assumption, as DIC as well as bright-field and phase constract imaging are very often used for determining cell morphology [TODO cite Kasprowicz 2017]

    This chapter provides a brief overview of biological background knowledge needed to understand the process of cell line development (CLD) and the role of fluorescent \textit{in silico} labeling of DIC cell imaging within it. It also covers the fundamentals of deep and machine learning techniques used in this work including clustering and dimentionality reduction approaches. At the end of the chapter a brief summary of the microscopy image acquisition used in the research is outlined.
    \subsection{Biology}
        \subsubsection{Cell line development process}
        The cell line development (CLD) is a process of generating single cell-derived clones that produce high and consistent levels of target therapeutic protein (\cite{lonza}). Therapeutic proteins in this case are so-called recombinant proteins and they are widely used in biomedical research, the production of medication and for various therapeutic needs such as, for example, vaccines and monoclonal antibodies (mAbs) (\cite{Ohtake_2013}, \cite{Jefferis_2017}, \cite{Funaro_1996}). A recombinant protein, as defined by \cite{Barbeau_2018}, is a modified or manipulated protein encoded by a recombinant DNA. Recombinant DNA in turn consists of a plasmid, where the genes of the target protein of interest are cloned downstream of a promoter region. As soon as this plasmid is transfected to a host cell (for example some mammalian cells that are able to produce the protein), the host will start to express this protein of interest. Today there is a great need for the production of high volumes of good quality recombinant proteins, both in industrial as well as research contexts (\cite{Tihanyi_2020}). For this reason the goal of many research projects in recombinant protein production is to improve expression efficiency and create high-throughput systems to improve the CLD processes (\cite{Tihanyi_2020}).

% TODO \cite[see]{IWNLP} \supercite{iqbal2007underwater}

% TODO [cite Beckman] was for "remain the most popular choice"

One of the most popular host cells used in CLD and in this thesis specifically are chinese hamster ovary (CHO) cells (\cite{Castan_2018}). Although different cells can be used as hosts, such as bacterial, plant-based or yeast cells, mammalian cells remain the most popular choice. The reason behind this popularity resides in the fact that they can produce a diverse range of correctly folded proteins and most importantly they have high protein production rates. The productivity rate is measured in titre of produced protein, and CHO cells can reach 0.1 - 1 g/L in batch and 1 - 10 g/L in fed-batch cultures (\cite{Tihanyi_2020}). Mostly all of the mAbs are produced using CHO cells (\cite{Lalonde_2017}). Pharmaceutical companies very often try to use the same host cell line for all their productions because already checked and qualified cells simplify the road to the clinic (\cite{Tihanyi_2020}). Since this research is dedicated to CHO cell line as well, it has a wide applicability.

However, there is a downside to using CHO as host cells ---- these rapidly growing immortal cells are also genomically unstable and extremely heterogeneous which usually leads to the main issue: production instability. Instability here means that the cell might die or do not produce target protein. The problem of choosing stable and high-production clones that simultaneously will be able to express protein qualitatively and quantifiably over time is essentially the main goal of current research. The challenge in manufacturing here is the time and the cost of production. Currently, a lot of research attention is dedicated to the reduction of both factors, as well as the development of techniques of high-throughput clone screening and characterization (\cite{Tihanyi_2020}). The latter is of interest for this thesis. With great amounts of data collected over time and the development of computational modelling and statistical analysis it is now possible to carry out the analysis \textit{in silico}, meaning computationally without interfering with the cells instead of the usual \textit{in vitro} analysis (\cite{Christiansen_2018}), which will be also shown in this research.

\paragraph{CLD steps}
\label{section:cld-steps}
\begin{figure}[H]
	\begin{center}
		\includegraphics[width=0.8\linewidth]{bilder/CLD.png}
		\caption[CLD process steps]%
		{CLD process steps. Usual times needed for this process: from transfection to characterization --- 5 months, stability --- 3 months. Minipool selection step that is optimized here is taking up to 5 weeks.}\label{fig:cls-steps}
	\end{center}
\end{figure}

The first step of CLD is called transfection --- the introduction of the gene of interest (abbreviated as GOI or can be called a DNA vector or, alternatively, an expression vector) into CHO cells. There are two main challenges with this step: firstly, transfection mostly results in a vector being inserted into a random site within the host cell genome and secondly, it generally has low efficiency of integration  (\cite{Tihanyi_2020}). It is important to transfect a GOI into the optimal site of the genome to secure high protein expression over time during protein production. Practically however, GOI is transfected into a random location of the genome. In cases where the gene was transfected into an inactive site of genome (essentially the majority of genome is transcriptionally inactive), the cell will likely be unable to express the gene (\cite{Castan_2018}, \cite{Hong_2018}).

%TODO [a better reference needed Shin 2020]
The second step of the process is the selection of cell minipools that have successful and stable gene integrations for further expansion and cloning. The reason for not all of them being suitable is that during the transfection step, only 80\% of the cells will receive a GOI vector (\cite{Castan_2018}). Only a small percentage of these cells actually integrate a vector into the genome and, as mentioned above, only a fraction of those are able to express the protein in a stable fashion (\cite{Shin_2020}). After the best minipools are selected, they will be expanded, which means that cell population is serially passaged to a larger population with the bigger number of cells.

The third step in CLD is to clone the cells. The selected stable pools of cells are phenotypically and genetically diverse. This means that they have different growth rates, metabolic profiles, and so forth. This is not ideal for industrial production - all the cells used for protein production should be derived from the same clone (\cite{ema_2020}). 

Once the cells are cloned, phenotypical and genetical heterogeneity is reduced, the next step is to characterize the cells for their expression of the GOI. One has to estimate the clones' productivity and stability. Such observations may take up to 90 days (usually stability measurements are made on the $30^{\text{th}}$, $60^{\text{th}}$ and $90^{\text{th}}$ days). If the clones remain stable after this time and are able to express enough of the protein, then they are suitable for further production. However, this last step is very time-consuming and requires maintenance for feeding and analysing the cells. Predicting productivity and stability of the cells in earlier stages would reduce this time significantly or even allow to avoid this process entirely.
        \subsubsection{Project specifications of cell line development for Merck KgaA}
        There are many different proteins that can be produced using such technologies, for example, vaccines, hormones, sugars etc., however this research is dedicated to the production of monoclonal antibodies (mAbs). 

CHOZN® Platform is a currently widely used product of Merk KgaA. CHOZN is a CHO mammalian cell expression system for fast and easy selection and growth of clones producing high levels of recombinant proteins [cite tech-bulletin]. The processes of developing expression systems on this platform correspond to the general CLD process described in the previous subsection [put subsection number]. The scope of the project is to simplify the labour-intensive and time-consuming process of stability measurement of the expression system by inducing predictions of productivity and stability rates during early steps in the CLD process. 

After the transfection step there are several quantities that are measured in minipools in order to select the best ones. For example, cell size, its complexity, cell surface protein expression, endoplasmic reticulum (ER) mass, mitochondria mass, etc. For qualitative and quantitative characterization of cells, fluorescent labeling is used. It is a process of covalently binding fluorescent dyes to biomolecules such as nucleic acids or proteins, so that they can be visualized via fluorescence imaging [cite https://www.nature.com/subjects/fluorescent-labelling]. A fluorophore is a chemical compound that can reemit light at a certain wavelength.These compounds are a critical tool in biology because they allow experimentators to capture particular components of a given cell in detail [cite O'reilly life sciences p113].

Unfortunately, fluorescence labeling is expensive, time-consuming and may kill the cell due to its phototoxicity [cite Fried et al., 1982; Patil et al., 2018; Progatzky et al., 2013)]. Additionally, Yeo et al. [cite Tihanyi] found out that different selection markers affect the production stability of CHO cells. Other negative aspects of manual staining appoach are: there is a limited number of available fluorescent channels in microscopes; some fluorophores have a spectral overlap, hence there is a limited number of detectable markers [cite Perfetto et al., 2004]; such labels can be inconsistent [cite Burry, 2011; Weigert et al., 1970), and depend a lot on reagent quality and require many hours of lab work. Toxicity, for instance, is a very dangerous factor, especially for medicine production as it may even affect the final product. Therefore there exists a need for an approach of \textit{in silico} flurescent labeling - computationally and without affecting the cell. 

For \textit{in silico} labeling, the input data is a differential interference contrast (DIC) microscopy. This is an optical microscopy technique used to enhance the contrast in unstained, transparent samples [cite wikipedia?]. This is a much cheaper image acquision technique than a staining process, and it has much less variability as well (for example, no dependency on the dye or antibody quality). The research is dedicated to predicting fluorescence signal from the DIC imaging directly without the need of actual cell staining. The measurements needed for selection of minipools can be calculated as usual, but using the predicted images instead.

    \subsection{Deep learning and machine learning basics}
        Introduction of the notaiton for the dataset, parameters, predictions.
        \subsubsection{Neural networks}
            \begin{definition}[Image dataset]
	An image dataset in the scope of this thesis constists of input DIC images $X$ and target fluorescence images $Y$. Combined, couples from each form (X and Y) construct a dataset:
	\begin{equation}
		D = (X, Y) = \{(x^{(1)}, y^{(1)}), \dots, (x^{(N)}, y^{(N)})\}
	\end{equation}

	where both $x^{(i)}$ and $y^{(i)} \in \mathbb{R}^{W \times H}$ are single images, $N$ is the size of the dataset. Generally input data has a shape of $(N, C, H, W)$, in this work $C = 1$.
\end{definition}

\begin{definition}[Model]
	A model is a function with learnable parameters $\theta = (\theta_1, ..., \theta_K)$ where $\theta_i \in \mathbb{R}$ for $i \in {0, ..., K}$ which approximates the mapping of initial data $X$ to target data $Y$.
	\begin{equation}
		M(X,\theta) = Y^\prime \approx Y 
	\end{equation}
\end{definition}

\begin{definition}[Loss function]
	A loss function is a function $L(y, M(x, \theta))$ of model's parameters $\theta$, that for $(x^{(i)}, y^{(i)}) \in D$ outputs a scalar value measuring the difference between ground truth $y$ and prediction $M(x, \theta)$. A training objective is then defined as an average over the loss of each training sample:
	\begin{equation}
		J(\theta) = \mathbb{E}_{(x, y)\sim p_{data}} L(y, M(x, \theta))
	\end{equation}
	where $p_{data}$ denotes an empirical distribution of the training data.
\end{definition}

\begin{definition}[Binary-cross entropy loss]
	Let $y \in \mathbb{R}^{W \times H}$ be a ground truth image and $y^\prime \in \mathbb{R}^{W \times H}$ be a prediction. Binary-cross entropy loss is defined as:
	\begin{equation}
		L(y, y^\prime) = - \frac{1}{N^2}\sum_{i=1}^{H} \sum_{j=1}^{W} y_{i,j} \cdot \log(y_{i, j}^\prime) +  (1 - y_{i, j}) \cdot \log(1 - y_{i, j}^\prime) 
	\end{equation}
\end{definition}

\begin{definition}[MSE (mean squared error) loss]
	Let $y \in \mathbb{R}^{W \times H}$ be the ground truth and $y^\prime \in \mathbb{R}^{W \times H}$ be the predicted images. The MSE loss is defined as:
	\begin{equation}
		L(y, y^\prime) = \sum_{i=1}^{H} \sum_{j=1}^{W} (y_{i, j} - y_{i, j}^\prime)^2
	\end{equation}
\end{definition}

\begin{definition}[PCC (Pearson correlation coefficient) loss]
	\label{def:pcc-loss}
	Let $y \in \mathbb{R}^{WH}$ be a flattened ground truth and $y^\prime \in \mathbb{R}^{WH}$ be a flattened predicted image. The PCC loss is defined as:
	\begin{align}
		PCC(y, y^\prime) &= \frac{\sum_{i=1}^{{WH}}{(y_i - \bar{y})(y_i^\prime - \bar{y}^\prime)}}{\sqrt{\sum_{i=1}^{{WH}^2}{(y_i - \bar{y})^2(y_i^\prime - \bar{y}^\prime)^2}}}  \\
		L(y, y^\prime) &= \frac{1 - PCC(y, y^\prime)}{2}
	\end{align}
	where $\bar{y}$, $\bar{y}^\prime$ are means of the ground truth and predicted images respectively.
	
	There is an important distinction to be made here: firstly, Pearson correlation coefficient (PCC further) in a measure of similarity between two data sequences (matrices in this case), with values between $-1$ and $1$, with $1$ being a positive correlation, secondly, PCC loss is a measure of dissimilarity between two matrices, with values between $0$ and $1$, with $0$ meaning that matrices are the same.

	This loss is widely used in cell biology for comparison of co-localization between the proteins (\cite{Lachance_2020}). PCC is also popular in computer vision where it is utilized for the determination of image similarity in terms of spatial-intensity (\cite{Lachance_2020}).
\end{definition}

\begin{definition}[Optimization]
	Optimization is a process of updating the parameters $\theta$ of the model $M(X, \theta)$ to minimize the loss function $L(y, M(x, \theta))$.
\end{definition}

With a maximum likelihood esimation, we get:
\begin{equation}
	\theta_{MLE} = \argmax\limits_{\theta} \sum_{i=1}^{N} \log{p_{\text{model}}(x^{(i)}, y^{(i)}, \theta)}
\end{equation}

After maximizing the sum and taking a gradient one gets:
\begin{equation}
	\nabla_{\theta} J(\theta) = \mathbb{E}_{x, y \sim p_{data}} \nabla_{\theta} \log{p_{\text{model}}(x, y, \theta)}
\end{equation}

The exact gradient on a discretized data-generating distribution is then:
\begin{equation}
	g = \nabla_{\theta} J^*(\theta) = \sum_{x} \sum_{y}{p_{\text{data}}(x, y) \nabla_{\theta} L(y, M(x, \theta))}
\end{equation}

Here one can obtain an unbiased estimator of a true gradient on a mini-batch of i.i.d. samples $\{x^{(i)}, ..., x^{(m)}\}$	

\begin{equation}
	\hat{g} = \frac{1}{m} \nabla_\theta \sum_{i} L(y^{(i)}, M(x^{(i)}, \theta))
\end{equation}

\begin{definition}[Stochastic gradient descent]
	Stochastic gradient descent is an optimization algorithm where the parameters $\theta$ are iteratively updated every mini-batch of data by the following rule:
	\begin{equation}
		\theta_{k+1} = \theta_k - \alpha \frac{1}{m} \nabla_\theta \sum_{i} L(y^{(i)}, M(x^{(i)}, \theta))
	\end{equation}
	where $\alpha$ is a tuneable parameter called \textit {learning rate}.
\end{definition}

\begin{definition}[Adadelta optimizer]
	An Adadelta optimizer is a more sophisticated optimization technique, that follows algorithm \ref{alg:adadelta} for the parameter update.
	\begin{algorithm}[H]
		\caption{Adadelta optimization}\label{alg:adadelta}
		\item 1. $E[g]^2_0 = 0$ and $E[\Delta \theta^2]_0 = 0$
		In order to update the parameters one needs to:
		\item 2. Compute gradient: $g_t$
		\item 3. Accumulate gradient: $E[g]^2_t = \rho E[g]^2_{t - 1} + (1 - \rho)g_t^2$
		\item 4. Compute update: $\Delta \theta_t = \frac{\text{RMS}[\Delta \theta]_{t-1}}{\text{RMS}[g]_t} \hat{g_t}$
		\item 5. Accumulate updates: $E[\Delta \theta^2]_t = \rho E[\Delta \theta^2]_{t-1} + (1 - \rho) \Delta \theta^2_t$
		\item 6. Apply update: $\theta_{t+1} = \theta_t + \Delta \theta_t$ \\
		RMS here is the root mean square all initial hyperparametes are take from the original study(\cite{Zeiler_2012}).
	\end{algorithm}
\end{definition}

\begin{definition}[Adam optimizer]
	An Adam optimizer is another stohastic optimization technique, that has the following hyperparameters: $\alpha$ --- learning rate, $\beta_1, \beta_2 \in [0, 1)$ --- exponential decay rates. It follows algorithm \ref{alg:adam} for the parameter update.
	\begin{algorithm}[H]
		\caption{Adam optimization}\label{alg:adam}
		\item 1. Initialize: $m_0 = 0$ and $v_0 = 0$
		\item 2. Compute gradient: $g_t$
		\item 3. Update biased first moment estimate: $m_t = \beta_1 m_{t-1} + (1 - \beta_1) g_t$
		\item 4. Update biased second raw moment estimate: $v_t = \beta_2 v_{t-1} + (1 - \beta_2) g_t^2$
		\item 5. Compute bias corrected first moment estimate: $\hat{m_t} = \frac{m_t}{1 - \beta_1^t}$
		\item 6. Compute bias corrected second raw moment estimate: $\hat{v_t} = \frac{v_t}{1 - \beta_2^t}$
		\item 7. Apply update: $\theta_{t+1} = \theta_t - \alpha \frac{\hat{m_t}}{\sqrt{\hat{v_t} + \epsilon}}$
		Initial hyperparameters used in this work are $\alpha = 0.001$, $\beta_1 = 0.9$, $\beta_2 = 0.999$ and $\epsilon = 10^{-8}$.
	\end{algorithm}
\end{definition}

\begin{definition}[Overfitting]
	Overfitting is a phenomenon in which a hypothesis that fits training samples well will perform worse over the entire distribution on data rather than another hypothesis that fits the distribution of the training samples less well (\cite{mitchell_1997}). The way to avoid overfitting that happened to the models in Section \ref{section:er} are discussed in Section \ref{section:regularization}.
\end{definition}

\begin{definition}[Feedforward fully connected layer]
	A feedforward fully connected layer is a trainable function with parameters $W \in \mathbb{R}^{N \times M}$ (weights) and $b \in \mathbb{R}^{M}$ (biases) that, in this case, maps a vector $x \in \mathbb{R}^{N}$ to an output $a \in \mathbb{R}^{M}$ via the following transformation:
		\begin{equation}
			a = W^{T}x + b
		\end{equation}
\end{definition}

This is one of the simplest layers in a feedforward neural networks and input and output in it as mentioned above are vectors. However, in this study inputs and outputs are images, that are represented in memory as square matrices $x^{(i)}, y^{(i)} \in \mathbb{R}^{N \times N}$. One could simply flatten the image into a vector and use it as an input to a fully connected feedforward neural network. Nevertheless this would be a suboptimal approach. 

Since essentially one of the main tasks of this research is to create a deep learning model that is able to predict a fluorescence image from a DIC image, the problem statement could be narrowed down to the following: predict an intensity high-resolution image from another intensity high-resolution image based on the features of the object morphology in it. Such problem is very common in the field of image analysis and one of the popular deep learning tools for solving such problems is convolutional neural network (CNN) or more specifically a UNet.

CNNs are able to capture nonlinear relationships over large areas of images, they greatly improve performance for image recognition tasks in comparison to classical machine learning methods (\cite{Ounkomol_2018}). The word "convolutional" suggests that the convolution operation should be used in at least one of the layers there.  

\begin{definition}[Convolutional layer]
	A convolutional layer is a trainable function with parametrized kernel $K \in \mathbb{R}^{F \times F \times C}$ and bias $b \in \mathbb{R}$ that is usually denoted via the operator $(\cdot * \cdot)$. By transforming an input $x \in \mathbb{R}^{W \times H \times C}$ it produces an output $S$
	\begin{equation}
		S = K * x + b
	\end{equation}

	that is called a \textit{feature map} where an element on position $(i, j)$ is defined as follows:
		\begin{equation}
			S_{i, j} = \sum_{w} \sum_{h} x_{m, n}  K_{i - m, j - n}
		\end{equation}
\end{definition}

Convolutional layer like a fully connected layer can be viewed a linear transformation as well. However, there are three main advantages that leverage convolutional layers for image processing in comparison to fully connected layers: sparse interactions, parameter sharing and equivariant representations. An image is a very redundant way of representing the semantic meaning hidden within it. Having a value of one pixel, the probability that the neighboring one will be of the same color is very high. Sparsity of interactions can be described by an example: usually a high-resolution image might have millions of pixels, however it is possible to detect smaller and very important features like contrast changes, edges, and shapes using a kernel consisting of only a few hundred pixels. By applying kernels (or filters) on the image locally, one will infer many of these features across the whole image. Such an approach reduces the memory needed for parameter storing and improves its statistical efficiency (\cite{Goodfellow_2016}). Parameter sharing refers to the fact that instead of learning a separate set of parameters for every location within the image, only one set of parameters will be learned and applied across all image locations. Lastly, equivariance here means that convolution operation is equivarient to the shifts in the image.

\begin{definition}[Stride]
	During the computation of convolution, the kernel starts sliding at the upper left corner of the input tensor, covering all locations while heading to the right and down. The step with which the window slides is called \textit{stride}. 
\end{definition}

\begin{definition}[Padding]
	When convolution is applied several points on the perimeter of the input tensor will be lost and the ouput tensor will have smaller spatial dimension than the input one. One can fix this by adding a few more pixels outside the perimeter, to preserve the dimension of the output to be same as input. The amount of pixels added is called \textit{padding}. 
\end{definition}

Visual examples of what stride and padding represent are shown in Figure \ref{fig:stride}.
\begin{figure}[htb]
	\begin{center}
		\includegraphics[width=0.6\linewidth]{bilder/stride_padding.png}
		\caption[Stride and padding example]%
		{Stride and padding example. Taken from \cite{stride}.}
		\label{fig:stride}
	\end{center}
\end{figure}

\begin{definition}[Max-pooling layer]
	Maximum pooling operation reports the maximum output within a rectangular neighborhood (\cite{Goodfellow_2016}).
\end{definition}

\begin{definition}[Activation function]
	An activation function is an element-wise non-linear function $f(\cdot)$. Some examples are:
	\begin{align}             
		f(x) = \frac{1}{1 + e^{-x}} &&\text{Sigmoid} \\      
		f(x) = max(0, x) &&\text{Rectified linear unit (ReLU)}\\
		f(x) = \begin{cases}
				x, \hspace*{1cm} \textrm{if } x > 0 \\
				\alpha * (e^{x} - 1), \textrm{if }  x \leq 0
		  	\end{cases}\ &&\text{ELU}
		\end{align}
\end{definition}

It is important to use activation functions after each convolutional or linear layer like RELU, ELU, Tahn, Sigmoid or any other non-linearities. Because any combination of linear functions can be represented with another linear function, having consecutive linear layers without non-linear function in the network is equivalent to having just one linear layer. Non-linearities  In CNNs they are also often combined with max-pooling layers and dropouts to escape overfitting. 

\begin{definition}[Batch normalization layer]
	Let's denote $B = \{x^{(i)}, ..., x^{(m)}\}$ to be a mini-batch of data. Then batch normalizing transform applied to this input data would be:
	\begin{equation}
		\begin{split}
		& a^{(i)} = \gamma \frac{x^{(i)} - \mu_B}{\sigma^2_B + \epsilon} + \beta \\
		& \sigma^2_B = \frac{1}{m} \sum_i^m (x^{(i)} - \mu_B)^2 \\
		& \mu_B = \frac{1}{m} \sum_i^m x^{(i)} \\
		\end{split}
	\end{equation}
	where $\gamma$ and $\beta$ are learnable parameters, $\mu_B$ and $\sigma^2_B$ are the mean and standard deviation of the batch (\cite{Ioffe_2015}).
\end{definition}

\begin{definition}[Dropout layer]
	Dropout is a technique that randomly sets some weights (units) to zero (\cite{Srivastava_2014}). It leads to the training of several smaller networks that share the parameters. If a mask vector $\mu$ specifies which units are included in training, then dropout's objective to be minimized becomes: $\mathbb{E}_\mu J(\theta, \mu)$. Visually dropout is presented in the Figure \ref{fig:dropout}.
\end{definition}

%TODO add figure reference!
\begin{figure}[H]
	\begin{center}
		\includegraphics[width=0.5\linewidth]{bilder/dropout.png}
		\caption[Dropout]%
		{Dropout. Taken from \cite{Srivastava_2014}}
		\label{fig:dropout}
	\end{center}
\end{figure}

The models in this project mostly use ELU activations as ELU provides a better signal flow between the layers by not cutting off the negative values completely.

\begin{definition}[UNet]
	UNet is fully convolutional neural network with U-shaped encoder-decoder network architecture (\cite{Ronneberger_2015}). Example of the UNet architecture can be found in Figure \ref{fig:unet}.
\end{definition}

The encoder is a common CNN, consisting of the repeated
block of two $3 \times 3$ convolutions, followed by
an activation function, and a $2 \times 2$ max-pooling operation with stride 2. At each encoder step  the number of feature channels doubles. The decoder is also a CNN, consisting of repeated blocks of transposed convolution, that halves the number of feature channels, followed by a concatenation with a corresponding output from an encoder, and two $3 \times 3$ convolutions, followed by a ReLU. The last decoder layer is a $1 \times 1$ convolution to map the tensor to the number of output image channels needed. Skip-connections is a very important part of UNet as they allow to the flow of high-resolution features from the encoder to the decoder that in turn allows to restore a corresponding high-resolution image.

\begin{definition}[Autoencoder]
	Autoencoder is an unsupervised learning technique in neural networks for the representation learning purposes. Autoencoder consists of an encoder that compresses data into a lower dimensional representation and a decoder that restores the original input from the encoded representation.
\end{definition}

\paragraph{Regularization techniques}
\label{section:regularization-theory}
% TODO add overfitting image?
Regularization is mostly used to prevent a deep learning model to overfitting on the training data and to be able to generalize well. Overfitting has occured in the models used in this research and therefore it is improtant to understand the techniques that can be used to prevent it. There are are several approaches to regularize the model and they will be explained below.

\begin{itemize}
	\item Early-stopping

	Overfitting can be detected via visualizing train and validation losses. Training behaviour at first will be the usual one, meaning that both train and validation losses are gradually decreasing, however at some point the train loss continues to decrease, whereas the validation loss suddenly starts to increase. Since the model has not seen any of the data from the validation set, it means that it loses its ability to generalize on unseen data, while improving its perfomance on the seen data (train set). This does not happen during earlier epochs. Assuming that the model learns a complex decision surface while training, the weights of the model will be quite small and random with the correct weight initialization and therefore the best decision surface during the early epochs would be a smooth one. But during the later ones the difference in values of the weights grows and they become dissimilar which also means that the decision surface becomes more complex and the model is now able to fit not only the training data itself, but also its noise (\cite{mitchell_1997} p.111). And that is why stopping before the model becomes too complex, meaning to stop before the overfitting point, mitigates this problem.

	\item \emph{L1}- \emph{L2}-regularization

	The complexity of the deep model grows with the number of features it uses, sometimes the model may pay attention to the features that are not important to the outcome, or even considers noise to be a feature. To prevent this one should decrease the weights associated with useless features, however one cannot know ahead of time which of them should be ignored, therefore one may limit them all (\cite{Ying_2019}). In order to do that, a penalty term is added to the loss function:

	\begin{equation}
	\tilde{L}(Y, M(X, \theta)) = L(Y, M(X, \theta)) + \lambda R(\theta)
	\end{equation}

	for some $\lambda > 0$. This is called a \emph{soft-constraint} optimization. When $R(\theta)$ is of the form $R(\theta) = ||\theta||^2_2 = \sqrt{\sum\limits_i \theta_i^2}$ this is called \emph{L2}-regularization. When it is of form $R(\theta) = ||\theta||_1 = \sum\limits_i |\theta_i|$ this is called \emph{L1}-regularization. \emph{L2}-regularization used in combination with backpropagation is equivalent to weight decay. Weight decay is defined by \cite{Hanson_1988} as follows:
	\begin{equation}
		\theta_{t+1} = (1 - \lambda)\theta_t - \alpha \frac{\partial L}{\partial \theta_t}
	\end{equation}

	where $\alpha$ is a learning rate. Weight decay successfully has more effect on the weights along which the gradient change is smaller \cite{Goodfellow_2016}. \emph{L1}-regularization induces sparsity of the weights by assigning some of them to zero, this could also be considered as a feature selection approach.

	\item Regularization layers

	Batch normalization and dropout layers are also considered to be a form of regularization.

	\item Network reduction

	Since learning a too complex and noise-fitting decision surface might be a frequent cause of an overfit, another way to mitigate this would to be reduce the space of the possible decision surfaces and therefore make the surface simpler so that it cannot fit into the noise from the data. By changing the number of adaptive parameters in the network, the complexity can be varied (\cite{Bishop_2006} p.332).

	\item Expansion of the training data

	For a successful training a model needs to have a sufficient amount of quality samples. An expanded dataset can improve the quality of the predictions \cite{Ying_2019}, however only when the model has already performed well on the initial dataset. If the model was performing badly initially, adding more data will not solve the problem.
\end{itemize}

        \subsubsection{Dimensionality reduction methods}
            \paragraph{UMAP}
\paragraph{PCA}
\paragraph{PacMAP}
        \subsubsection{Clustering methods}
            After visualizing the embeddings, there is an interest in checking whether they form any kind of clusters. This question is discussed in Section \ref{section:unet-embeddings-dim-reduction} using the DBSCAN algorithm. This part will provide the theory needed to understand how this algorithm works and how it can be set up.
\paragraph{DBSCAN}
\label{section:dbscan}
The DBSCAN is a density-based unsupervised algorithm for discovering clusters. It considers regions with high-density of points to be clusters and points that are located far away from any cluster or form a low density region to be ourliers. This algorithm uses two hyperparameters = \{\textit{eps}, \textit{min\_samples}\} to define the clusters. The first is the distance threshold, which is used to determine whether a point is located in the neighborhood of the other point. The latter one is the minimum number of points that are needed to form one cluster. It splits the points into four categories based on these hyperparameters: \textit{core point} (\textit{min\_samples} points are reachable from it), \textit{directly reachable point} (is within distance \textit{eps} from any core point),  \textit{reachable} (there is a path from a core point to it via directly reachable points) and \textit{outliers}. For more information on how clusters are formed refer to \cite{dbscan}. DBSCAN does not require the provision of the number of clusters in advance. Although this is a nice quality of this algorithm it is not that important for current research as the number of clusters that is needed here is known in advance. For example, this algorithm is used in Section \ref{section:unet-embeddings-study} in order to check whether different phenotypes form different clusters or, for example, whether corrupted images would fall into a separate clusters. In all cases the number of ground truth clusters is known in advance.  
    \subsection{Imaging}
        \subsubsection{Digital imaging}
            Digitally an image is represented as an array of size $(H, W, C)$ where $H$ is the height, $W$ is the width and $C$ is the number of channels of the image. In this work, $C = 1$ and $W = H$. A digital image $A$ can be represented represented with the matrix:

\begin{equation}
    A = \left[
            \begin{array}{ccc}
                a_{0,0} & \cdots & a_{0,W-1} \\
                \vdots & \ddots & \vdots \\
                a_{H, 0} & \cdots & a_{H-1, W-1}
            \end{array}
        \right]
\end{equation}
where $a_{i, j} \in \mathbb{R}$. Both DIC and fluorescence images were provided in tag image file format (TIFF). For the processing convenience purpose all images were normalized to be in the range of $[0, 1]$:

\begin{equation}
    a^{\text{norm}}_{i,j} = \frac{a_{i, j} - \min(A)}{\max(A) - \min(A)}
\end{equation}
for $ \forall i \in \{0, ..., W - 1\}$ and $ \forall j \in \{0, ..., H - 1\}$
            How image is stored in memory, which conventions there are (RGB, BGR (conventions are used in corruptions augmentations)).
        \subsubsection{Microscopy imaging}
            \paragraph{Image acquisition peculiarities} 
    Cells used in this research are growing in 96-well plates. A plate or a microplate in biology is a flat plat with multiple tubes ("wells"). The microscope used in the experiments takes photos of the well plate in random locations. The reason for that hides in the settings for focuing in microscope. To get a reasonably good photo without blur it has to focus on a specific location of the plate, the choice of the location however happens automatically, therefore the location of the focus is random (see Figure \ref{fig:random-dic}). 
    
    Unfortunately it might be problematic in the following sense: photos takes in such manner do not gurantee that the focus will land in distinct spots all the time. Meaning that some cells present in one of the photos might appear in the other ones. Since the photos have a high-resolution they are first splitted into crops of size $256 \times 256$ each. It might happen that same cells might appear in several crops. That is why after split of the image data between train, test and validation sets it might be that the same set of cells will once land in the train set and another time in the validation set, which will lead to a not completely fair and representative validaiton loss during training.
    
    In order to overcome this problem a much more expensive equipment is needed. Since in our case it doesn't bring too huge problems expect for the fact that validation metrics might be lower than they should have been, there was no need to purchase a more expensive equipment.   
    
    \begin{figure}[htb]
        \begin{center}
            \includegraphics[width=0.3\linewidth]{bilder/dic-random.png}
            \caption{Way in which photos of the well-plate were taken}\label{fig:random-dic}
        \end{center}
    \end{figure}    
\paragraph{Crops combination technique}
    Due to the restricted amount of memory on GPU deep learning models cannot have a high-resolution image as their input within current research. Yet this is also not obligatory: as the image contains dozens of cells within it, its processing can be limited to a crop of a smaller size. After the model has predicted fluorescence signal for each of the crops, output fluorescence images can be combined together to form a high-resolution image again. In this thesis the architecture of the model assumes an input of size $(256, 256)$ or more specifically $(None, 1, 256, 256)$, where the first dimension is responsible for the batch size and the second one states that the input is a 1-channel image. 

There are several ways of how one can split the image, the easiest approach would be to use a sliding window of size $w$. This algorithm is depicted in the Figure \ref{fig:sliding-window}. A small window starts sliding the image from the left upper to the right down corner with step size $s$ feeding the selected crops into a deep learning model. From the output of the model only a center part of such a crop is accepted to form a full fluorescence image. Border size $b$ in this case is the size of the edges of the crop that are not accepted from the predictions of the deep learning model.

Having a desired border size, in order to accepted areas to be overlaped with each other without blank spaces, one has to adjust the step size to be:
\begin{equation}
  s = w - 2 * b
\end{equation}
When step size $s$ is equal to window size $w$, there is no overlap between the windows.

\begin{figure}[H]
	\begin{center}
		\includegraphics[width=0.3\linewidth]{bilder/sliding-window.png}
		\caption{Sliding window approach for fluorescence prediction}\label{fig:sliding-window}
	\end{center}
\end{figure}
The reason why not the full prediction is accepted to form the output hides in the following: trained models are less accurate on the borders of the crops rather than in the center. Most of the times there are cells on the borders of the crops that were sliced and therefore it might be impossible to make a good prediction for them just due to the lack of input information. Therefore the step size has to be smaller than the window size, so that the windows are overlapping and for each prediction we use only the image center and are allowed to ingore predictions on the border (see the comparison between different border sizes in the Figure \ref{fig:crops-combination}). This is dicsucced in more detail in Section [TODO reference the section]. Such approach helps to reduce the effect of the grid visible on the image composed of many small crops, which one can see in the Figure \ref{fig:crops-combination} on the left to almost non-visible borders as in the same Figure on right. This would of course take longer time in predictions, however as the speed is less crucial in comparison to the accuracy of the predictions.

\begin{figure}[htb]
	\begin{center}
		\includegraphics[width=\linewidth]{bilder/crops_combination/crops-combination.png}
		\caption{Difference of overlap between predictions on the resulting image}\label{fig:crops-combination}
	\end{center}
\end{figure}