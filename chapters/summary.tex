\section{Summary}
In this master thesis three deep learning models were developed that are able to predict fluorescence signal from DIC image data for the following targets: nuclei, endoplasmic reticulum and full cell surface. Training experiments for Golgi apparatus target have shown that there is a lack of training data with high enough signal-to-noise ratio and further research is required. Three models mentioned above can successfully replace manual fluorescence staining procedure.

Before training the models, several preparation stages have been developed. First, the model's architecture was improved for faster learning with the use of batch normalization layers. Image augmentations such as rotations and scaling were added and their logic was improved through the use of the high-resolution original image. The model was tuned by choosing correct regularization and optimization algorithms (batch normalization and adadelta optimizer in this case). For each of the four targets corresponding image preprocessing  procedures were developed in order to improve data quality and address model's limitations before training (such as background-foreground class imbalance for instance). Advanced background removal and enhancement approaches were developed for the Golgi preprocessing target.

In each training procedure, the model has successfully converged. Intensity fluorescence predictions as well as binary segmentation (for the full cell surface target) were visualized. It was shown that the use of a bigger model and more data significantly improves the predictions (especially total and mean intensities show an immediate improvement in correlation coefficients). For a better models evaluation, more practical downstream metrics were proposed. These are organelle quantity, total intensity, mean intensity and area of the organelle of interest. Every model was evaluated based on these metrics, it was shown that not only a significant correlation with ground truth values in terms of Pearson and Spearman rank correlation coefficients is present, but the forms of distributions visualized with violin plots are very similar. All models have a similar downside in overpredicting total and mean intensities, however strong correlation betwen the values suggests that there is an absolute value shift in predictions that can be relatively easily fixed.

Evaluation of the models on these downstream metrics also required  development of the corresponding segmentation pipelines for every organelle in question. They were successfully built with the OpenCV and \textit{skimage} libraries. 

UNet embeddings were studied for the possible source of additional data insights like phenotype differences or corruptions and unsupervised clustering algorithms were applied. The study has shown that image embeddings are not clustered based on cell phenotype, however corrupted images indeed form a separate cluster in the embeddings space. Nevertheless, this cluster is not well-separable from the rest of the data and further research is required. Autoencoder embedding that was trained on the same data, was checked for the same clustering targets. The results have shown strong clustering based on the brightness of the crops, which is not significant for this research.

Finally, the stability of the models was studied. Two types of image corruptions were introduced: artificial corruptions via image processing and corrupted data from the wrong microscopy settings. It was shown that models are very stable against changes in contrast and brightness, which can be caused by an over- or underexposure. Although the models were very prone to errors with defocus blur corruption, it was shown that including artificial corruptions in augmentations immediately improves the results. A generalizability study of the model across cell phenotypes and across not fixed cells was carried out. Prediction on the image with corruptions created in the laboratory have shown that models are stable against errors in cell fixation process, however they are quite sensitive towards errors in the focus of the microscope. In order to detect corruprions, two drift detection algorithms were developed. The first one is based on the maximum mean discrepancy method, and the second one is an online version of the first one. They were tested on both corruption types and have shown strong ability for detecting drift in data with high ROC-AUC scores.
  
\subsection{Limitations}
The main limitation of this research is the need to fix the cells before taking a DIC image of them. Cell fixation is a preceding step before cell staining --- therefore all cells in datsets of DIC images were fixed. Since living and fixed cells look very different and models trained on fixed cells do not generalize well to not fixed ones, predictions can be done on DIC images only. Luckily, fixing cells is not a cumbersome lab procedure and is far easier than staining the cells, which is avoided with the help of \textit{in silico} fluorescence labeling. It is recommended to look into possibilities of transfer learning from fixed to living cells.  

Also, models developed here cannot generalize well on other cell types and are able to give good predictions for the cells used in this laboratory only. However, in future the developed product would aim to address specific need of each laboratory separately. Therefore the models will be trained with the goal to address the issues of one specific cell line. 


\subsection{Future research}
This research sets the ground for a variety of futher research ideas:
\begin{itemize}
    \item As Golgi apparatus model did not produce good enough predictions due to several reasons, such as low signal-to-noise ratio in input data, as well as an extremely strong class imbalance between the foreground and background in images, it is strongly recommended to continue research in this direction. For example, by choosing another target protein in staining procedure, apppying stronger noise reduction approaches like \cite{noise2void}, and incorporating image gradients in the loss function.
    \item UNet embeddings do indeed show a potential for detecting corruptions in crops, however it is not strong enough. It is highly recommended to incorporate embeddings of all crops from the same image into one point in the embedding space, for example, by averaging the embeddings. This might show a more significant clustering of corrupted data.
    \item Since autoencoder embeddings have shown a clear clustering based on the crop brightness, it would be beneficial to normalize the brightness across all crops first. This would allow an autoencoder to pay attention to other less distinctive features in the image.
    \item Due to the time constraints it was not possible to train very big models, however the improvement of predictions quality that happens after a model enlargement is very significant. Therefore it is recommended to train models with a bigger model size and more data.
    \item The lack of data in corruptions from the microscopy settings did not allow to test online drift detection algorithm. As this version is very promising to be used in practice it is recommended to use more data for its testing.
\end{itemize}
 